%!TEX program = xelatex
\documentclass[a4paper,11pt]{article}
\usepackage{fontspec, xunicode, xltxtra,paralist,amsmath,esint,color,amssymb,graphicx,float,array,booktabs,appendix,indentfirst,cases,bm}
\usepackage{algorithm,algorithmic}
\usepackage{galois}
\usepackage{amsthm}
\usepackage{xeCJK}
\usepackage{bm}
\usepackage[a4paper, inner=1cm, outer=1.5cm, top=2cm,bottom=3cm, bindingoffset=1cm]{geometry}
%%%%%%%%%%%%%%%%%%%%%%%%%%%%%%%%%%%%%%%%%%%%%%
\newcommand\V[1]{\mathrm{\bf{#1}}}
\newcommand\Tx[1]{\mathrm{#1}}
\newcommand\Se[1]{\mathcal{#1}}
\newcommand\Db[1]{\mathbb{#1}}
\newcommand{\RN}[1]{\textup{\uppercase\expandafter{\romannumeral#1}}}
\newcommand\Ept[1]{\Db{E}\left[ #1 \right]}
\newcommand\Sma[2]{\sum\limits_{#1}^{#2}}
\newcommand\Prd[2]{\prod\limits_{#1}^{#2}}
\newcommand\Bop[2]{\bigoplus\limits_{#1}^{#2}}
\newcommand\Nm[1]{\lvert\lvert #1\rvert\rvert}
\newcommand\MB[1]{\left[#1\right]}
\newcommand\LB[1]{\{#1\}}
\newcommand\SB[1]{\left(#1\right)}
\newcommand\Prob[1]{\Tx{Pr}{\MB{#1}}}
\newcommand\PR[2]{\Tx{Pr}_{#1}{\MB{#2}}}
\newcommand\Var[1]{\Tx{Var}\MB{#1}}
\newcommand\Tc[1]{\textcircled{#1}}
%%%%%%%%%%%%%%%%%%%%%%%%%%%%%%%%%%%%%%%%%%%%%%%
\newtheorem{theo}{Theorem}
\newtheorem{coro}{Corollary}
\newtheorem{lemma}{Lemma}
\newtheorem{exam}{Example}
\newtheorem{Defi}{Definition}
\newtheorem{rem}{Remark}
\renewcommand\qedsymbol{Q.E.D.}

\begin{document}
\title{毕设论文定理2.5及推论2.1证明修订}
\author{Siyi Yang}
\maketitle

\section{定理2.4证明}

原论文FIM矩阵最后一行最后一个元素打印错误,修改如下:
\begin{equation}
\begin{split}
\V{J}_{\Tx{e}}(\V{P},k)=
\begin{pmatrix}
\bm{\Sigma}_{0}+\V{B}_{0}\V{B}^H_{0} & -\V{B}_{0}\V{F}^H_{1} & & & \\
 -\V{F}_{1}\V{B}^H_{0}  &  \bm{\Sigma}_{1}+\V{B}_{1}\V{B}^H_{1}+\V{F}_{1}\V{F}^H_{1} & -\V{B}_{1}\V{F}^H_{2} & &\\
 & -\V{F}_{2}\V{B}^H_{1}  &  \bm{\Sigma}_{2}+\V{B}_{2}\V{B}^H_{2}+\V{F}_{2}\V{F}^H_{2} & &\\
 &&&\ddots&\\
 &&&& \textcolor{red}{\bm{\Sigma}_{k}+\V{F}_{k}\V{F}^H_{k}}
\end{pmatrix}.
\end{split}
\end{equation}
其余证明不变。

\section{定理2.5证明}
根据定理2.4结论:

\begin{equation}
\begin{split}
\Se{J}(k)=&\Tx{tr}\{\bm{\Sigma}_{0}^{-1}\V{B}_{0}\SB{\V{I}+\V{B}_{0}^H\bm{\Lambda}_0^{-1}\V{B}_{0}+\V{F}_{1}^H\bm{\Sigma}_{1}^{-1}\V{F}_{1}}^{-1}\V{F}_{1}^H\\
&\bm{\Sigma}_{1}^{-1}\V{B}_{1}\SB{\V{I}+\V{B}_{1}^H\bm{\Lambda}_{1}^{-1}\V{B}_{1}+\V{F}_{2}^H\bm{\Sigma}_{2}^{-1}\V{F}_{2}}^{-1}\V{F}_{2}^H\cdots\\
&\bm{\Sigma}_{k-1}^{-1}\V{B}_{k-1}\SB{\V{I}+\V{B}_{k-1}^H\bm{\Lambda}_{k-1}^{-1}\V{B}_{k-1}+\V{F}_{k}^H\bm{\Sigma}_{k}^{-1}\V{F}_{k}}^{-1}\V{B}_{k-1}^H\bm{\Sigma}_{k-1}^{-1}\\
&\cdots \V{F}_{2}\SB{\V{I}+\V{B}_{1}^H\bm{\Lambda}_{1}^{-1}\V{B}_{1}+\V{F}_{2}^H\bm{\Sigma}_{2}^{-1}\V{F}_{2}}^{-1}\V{B}_{1}^H\bm{\Sigma}_{1}^{-1}\\
&\V{F}_{1}\SB{\V{I}+\V{B}_{0}^H\bm{\Lambda}_0^{-1}\V{B}_{0}+\V{F}_{1}^H\bm{\Sigma}_{1}^{-1}\V{F}_{1}}^{-1}\V{B}_{0}^H\bm{\Sigma}_{0}^{-1}\}.
\end{split}
\label{Jek}
\end{equation}


第$k$层网络中去掉边$e$时(\ref{Jek})中$\bm{\Sigma}_{k,2}$为$\bm{\Sigma}_k$, 保留边$e$时变为$\bm{\Sigma}_{k,1}=\bm{\Sigma}_k+\V{a}\V{a}^H$ ($\V{a}=\V{a}_{i,j}$). 根据定理2.4中$\bm{\Lambda}_k$的递推式,$\bm{\Sigma}_k$对(\ref{Jek})中其它值不产生影响。

\begin{equation}
\begin{split}
\Se{J}_1=&\Tx{tr}\{\bm{\Sigma}_{0}^{-1}\V{B}_{0}\SB{\V{I}+\V{B}_{0}^H\bm{\Lambda}_0^{-1}\V{B}_{0}+\V{F}_{1}^H\bm{\Sigma}_{1}^{-1}\V{F}_{1}}^{-1}\V{F}_{1}^H\\
&\bm{\Sigma}_{1}^{-1}\V{B}_{1}\SB{\V{I}+\V{B}_{1}^H\bm{\Lambda}_{1}^{-1}\V{B}_{1}+\V{F}_{2}^H\bm{\Sigma}_{2}^{-1}\V{F}_{2}}^{-1}\V{F}_{2}^H\cdots\\
&\bm{\Sigma}_{k-1}^{-1}\V{B}_{k-1}\SB{\V{I}+\V{B}_{k-1}^H\bm{\Lambda}_{k-1}^{-1}\V{B}_{k-1}+\V{F}_{k}^H\textcolor{red}{\SB{\bm{\Sigma}_{k}+\V{a}\V{a}^H}^{-1}}\V{F}_{k}}^{-1}\V{B}_{k-1}^H\bm{\Sigma}_{k-1}^{-1}\\
&\cdots \V{F}_{2}\SB{\V{I}+\V{B}_{1}^H\bm{\Lambda}_{1}^{-1}\V{B}_{1}+\V{F}_{2}^H\bm{\Sigma}_{2}^{-1}\V{F}_{2}}^{-1}\V{B}_{1}^H\bm{\Sigma}_{1}^{-1}\\
&\V{F}_{1}\SB{\V{I}+\V{B}_{0}^H\bm{\Lambda}_0^{-1}\V{B}_{0}+\V{F}_{1}^H\bm{\Sigma}_{1}^{-1}\V{F}_{1}}^{-1}\V{B}_{0}^H\bm{\Sigma}_{0}^{-1}\}.
\end{split}
\end{equation}

\begin{equation}
\begin{split}
\Se{J}_2=&\Tx{tr}\{\bm{\Sigma}_{0}^{-1}\V{B}_{0}\SB{\V{I}+\V{B}_{0}^H\bm{\Lambda}_0^{-1}\V{B}_{0}+\V{F}_{1}^H\bm{\Sigma}_{1}^{-1}\V{F}_{1}}^{-1}\V{F}_{1}^H\\
&\bm{\Sigma}_{1}^{-1}\V{B}_{1}\SB{\V{I}+\V{B}_{1}^H\bm{\Lambda}_{1}^{-1}\V{B}_{1}+\V{F}_{2}^H\bm{\Sigma}_{2}^{-1}\V{F}_{2}}^{-1}\V{F}_{2}^H\cdots\\
&\bm{\Sigma}_{k-1}^{-1}\V{B}_{k-1}\SB{\V{I}+\V{B}_{k-1}^H\bm{\Lambda}_{k-1}^{-1}\V{B}_{k-1}+\V{F}_{k}^H\textcolor{red}{\bm{\Sigma}_{k}^{-1}}\V{F}_{k}}^{-1}\V{B}_{k-1}^H\bm{\Sigma}_{k-1}^{-1}\\
&\cdots \V{F}_{2}\SB{\V{I}+\V{B}_{1}^H\bm{\Lambda}_{1}^{-1}\V{B}_{1}+\V{F}_{2}^H\bm{\Sigma}_{2}^{-1}\V{F}_{2}}^{-1}\V{B}_{1}^H\bm{\Sigma}_{1}^{-1}\\
&\V{F}_{1}\SB{\V{I}+\V{B}_{0}^H\bm{\Lambda}_0^{-1}\V{B}_{0}+\V{F}_{1}^H\bm{\Sigma}_{1}^{-1}\V{F}_{1}}^{-1}\V{B}_{0}^H\bm{\Sigma}_{0}^{-1}\}.
\end{split}
\end{equation}

根据以上二式,

\begin{equation}
\begin{split}
&\Se{J}^{\Tx{SL}}(e,k)\\
=&\Se{J}_1-\Se{J}_2\\
=&\Tx{tr}\{\bm{\Sigma}_{0}^{-1}\V{B}_{0}\SB{\V{I}+\V{B}_{0}^H\bm{\Lambda}_0^{-1}\V{B}_{0}+\V{F}_{1}^H\bm{\Sigma}_{1}^{-1}\V{F}_{1}}^{-1}\V{F}_{1}^H\\
&\bm{\Sigma}_{1}^{-1}\V{B}_{1}\SB{\V{I}+\V{B}_{1}^H\bm{\Lambda}_{1}^{-1}\V{B}_{1}+\V{F}_{2}^H\bm{\Sigma}_{2}^{-1}\V{F}_{2}}^{-1}\V{F}_{2}^H\cdots\\
&\bm{\Sigma}_{k-1}^{-1}\V{B}_{k-1}\{\SB{\V{I}+\V{B}_{k-1}^H\bm{\Lambda}_{k-1}^{-1}\V{B}_{k-1}+\V{F}_{k}^H\textcolor{red}{\SB{\bm{\Sigma}_{k}+\V{a}\V{a}^H}^{-1}}\V{F}_{k}}^{-1}\\
&-\SB{\V{I}+\V{B}_{k-1}^H\bm{\Lambda}_{k-1}^{-1}\V{B}_{k-1}+\V{F}_{k}^H\textcolor{red}{\bm{\Sigma}_{k}^{-1}}\V{F}_{k}}^{-1}\}\V{B}_{k-1}^H\bm{\Sigma}_{k-1}^{-1}\\
&\cdots \V{F}_{2}\SB{\V{I}+\V{B}_{1}^H\bm{\Lambda}_{1}^{-1}\V{B}_{1}+\V{F}_{2}^H\bm{\Sigma}_{2}^{-1}\V{F}_{2}}^{-1}\V{B}_{1}^H\bm{\Sigma}_{1}^{-1}\\
&\V{F}_{1}\SB{\V{I}+\V{B}_{0}^H\bm{\Lambda}_0^{-1}\V{B}_{0}+\V{F}_{1}^H\bm{\Sigma}_{1}^{-1}\V{F}_{1}}^{-1}\V{B}_{0}^H\bm{\Sigma}_{0}^{-1}\}.\\
\end{split}
\label{JSL}
\end{equation}


\begin{equation}
\begin{split}
&\SB{\V{I}+\V{B}_{k-1}^H\bm{\Lambda}_{k-1}^{-1}\V{B}_{k-1}+\V{F}_{k}^H\textcolor{red}{\SB{\bm{\Sigma}_{k}+\V{a}\V{a}^H}^{-1}}\V{F}_{k}}^{-1}\\
=&\SB{\V{I}+\V{B}_{k-1}^H\bm{\Lambda}_{k-1}^{-1}\V{B}_{k-1}+\V{F}_{k}^H\textcolor{red}{\SB{\bm{\Sigma}_{k}^{-1} - \bm{\Sigma}_{k}^{-1} \V{a}\SB{\V{I}+\V{a}^H \bm{\Sigma}_{k}^{-1}  \V{a}}^{-1} \V{a}^H\bm{\Sigma}_{k}^{-1} }}\V{F}_{k}}^{-1}\\
=&\SB{\V{I}+\V{B}_{k-1}^H\bm{\Lambda}_{k-1}^{-1}\V{B}_{k-1}+\V{F}_{k}^H \bm{\Sigma}_{k}^{-1}\V{F}_{k} -\textcolor{red}{\V{F}_{k}^H\bm{\Sigma}_{k}^{-1} \V{a}\SB{\V{I}+\V{a}^H \bm{\Sigma}_{k}^{-1}  \V{a}}^{-1} \V{a}^H\bm{\Sigma}_{k}^{-1} \V{F}_{k}}}^{-1}\\
=&\SB{\V{I}+\V{B}_{k-1}^H\bm{\Lambda}_{k-1}^{-1}\V{B}_{k-1}+\V{F}_{k}^H \bm{\Sigma}_{k}^{-1}\V{F}_{k}}^{-1}\\
&+\SB{\V{I}+\V{B}_{k-1}^H\bm{\Lambda}_{k-1}^{-1}\V{B}_{k-1}+\V{F}_{k}^H \bm{\Sigma}_{k}^{-1}\V{F}_{k}}^{-1}\V{F}_{k}^H\bm{\Sigma}_{k}^{-1} \V{a}\\
&\SB{1+\V{a}^H \bm{\Sigma}_{k}^{-1}\V{a}-\V{a}^H\bm{\Sigma}_{k}^{-1} \V{F}_{k}\SB{\V{I}+\V{B}_{k-1}^H\bm{\Lambda}_{k-1}^{-1}\V{B}_{k-1}+\V{F}_{k}^H \bm{\Sigma}_{k}^{-1}\V{F}_{k}}^{-1} \V{F}_{k}^H\bm{\Sigma}_{k}^{-1} \V{a} }^{-1} \\
&\V{a}^H\bm{\Sigma}_{k}^{-1} \V{F}_{k}\SB{\V{I}+\V{B}_{k-1}^H\bm{\Lambda}_{k-1}^{-1}\V{B}_{k-1}+\V{F}_{k}^H \bm{\Sigma}_{k}^{-1}\V{F}_{k}}^{-1}\\
=&\SB{\V{I}+\V{B}_{k-1}^H\bm{\Lambda}_{k-1}^{-1}\V{B}_{k-1}+\V{F}_{k}^H \bm{\Sigma}_{k}^{-1}\V{F}_{k}}^{-1}\\
&+\SB{1+\V{a}^H \bm{\Sigma}_{k}^{-1}\V{a}-\V{a}^H\bm{\Sigma}_{k}^{-1} \V{F}_{k}\SB{\V{I}+\V{B}_{k-1}^H\bm{\Lambda}_{k-1}^{-1}\V{B}_{k-1}+\V{F}_{k}^H \bm{\Sigma}_{k}^{-1}\V{F}_{k}}^{-1} \V{F}_{k}^H\bm{\Sigma}_{k}^{-1} \V{a} }^{-1} \\
&\SB{\V{I}+\V{B}_{k-1}^H\bm{\Lambda}_{k-1}^{-1}\V{B}_{k-1}+\V{F}_{k}^H \bm{\Sigma}_{k}^{-1}\V{F}_{k}}^{-1}\V{F}_{k}^H\bm{\Sigma}_{k}^{-1} \V{a}\V{a}^H\bm{\Sigma}_{k}^{-1} \V{F}_{k}\SB{\V{I}+\V{B}_{k-1}^H\bm{\Lambda}_{k-1}^{-1}\V{B}_{k-1}+\V{F}_{k}^H \bm{\Sigma}_{k}^{-1}\V{F}_{k}}^{-1}.\\
\end{split}
\end{equation}

代入(\ref{JSL})中得:

\begin{equation}
\begin{split}
&\Se{J}^{\Tx{SL}}(e,k)\\
=&\textcolor{red}{\SB{1+\V{a}^H \bm{\Sigma}_{k}^{-1}\V{a}-\V{a}^H\bm{\Sigma}_{k}^{-1} \V{F}_{k}\SB{\V{I}+\V{B}_{k-1}^H\bm{\Lambda}_{k-1}^{-1}\V{B}_{k-1}+\V{F}_{k}^H \bm{\Sigma}_{k}^{-1}\V{F}_{k}}^{-1} \V{F}_{k}^H\bm{\Sigma}_{k}^{-1} \V{a} }^{-1} }\\
&\lvert\lvert\bm{\Sigma}_{0}^{-1}\V{B}_{0}\SB{\V{I}+\V{B}_{0}^H\bm{\Lambda}_0^{-1}\V{B}_{0}+\V{F}_{1}^H\bm{\Sigma}_{1}^{-1}\V{F}_{1}}^{-1}\V{F}_{1}^H\\
&\bm{\Sigma}_{1}^{-1}\V{B}_{1}\SB{\V{I}+\V{B}_{1}^H\bm{\Lambda}_{1}^{-1}\V{B}_{1}+\V{F}_{2}^H\bm{\Sigma}_{2}^{-1}\V{F}_{2}}^{-1}\V{F}_{2}^H\cdots\\
&\bm{\Sigma}_{k-1}^{-1}\V{B}_{k-1}\SB{\V{I}+\V{B}_{k-1}^H\bm{\Lambda}_{k-1}^{-1}\V{B}_{k-1}+\V{F}_{k}^H \bm{\Sigma}_{k}^{-1}\V{F}_{k}}^{-1}\V{F}_{k}^H\bm{\Sigma}_{k}^{-1} \V{a}\rvert\rvert\\
\end{split}
\end{equation}

\section{推论2.1证明}

首先证明一个引理:
\begin{lemma}
\begin{equation}
\begin{split}
\Nm{\V{U}(\V{M}-\V{U}^H\V{U})^{-1}\V{U}^H}&=\frac{\Nm{\V{U}\V{M}^{-1}\V{U}^H}}{1-\Nm{\V{U}\V{M}^{-1}\V{U}^H}}\\
\Nm{\V{U}(\V{M}+\V{U}^H\V{U})^{-1}\V{U}^H}&=\frac{\Nm{\V{U}\V{M}^{-1}\V{U}^H}}{1+\Nm{\V{U}\V{M}^{-1}\V{U}^H}}.
\end{split}
\label{lemma1}
\end{equation}
\end{lemma}

\begin{proof}
\begin{equation}
\begin{split}
&\SB{\V{I}-\V{U}\V{M}^{-1}\V{U}^H}^{-1}=\V{I}+\V{U}(\V{M}-\V{U}^H\V{U})^{-1}\V{U}^H\\
\Longleftrightarrow &\V{U}(\V{M}-\V{U}^H\V{U})^{-1}\V{U}^H=\SB{\V{I}-\V{U}\V{M}^{-1}\V{U}^H}^{-1}-\V{I}\\
\Longrightarrow &\Nm{\V{U}(\V{M}-\V{U}^H\V{U})^{-1}\V{U}^H}=\Nm{\SB{\V{I}-\V{U}\V{M}^{-1}\V{U}^H}^{-1}-\V{I}}\\
\Longrightarrow &\Nm{\V{U}(\V{M}-\V{U}^H\V{U})^{-1}\V{U}^H}=\frac{\Nm{\V{U}\V{M}^{-1}\V{U}^H}}{1-\Nm{\V{U}\V{M}^{-1}\V{U}^H}}.\\
\end{split}
\end{equation}
注意到我们目前为止关注的矩阵范数中的矩阵都是对称且正定的,因此lemma1中的$M$应是对称正定的,$M-UU^H$其实也是(从11中对lemma1的应用可以看出)。在这个前提下,式(9)中的$UMU^H$是可以正交对角化的,假设为$QDQ^H$。
此时,式(9)右边变为$Q((I-D)^{-1}-I)Q^H$的模,也就是对角阵$(I-D)^{-1}-I$中的最大元素,因而式(9)是成立的(这里本来应该限制$UMU^H$的矩阵范数。
\begin{equation}
\begin{split}
&\SB{\V{I}+\V{U}\V{M}^{-1}\V{U}^H}^{-1}=\V{I}-\V{U}(\V{M}+\V{U}^H\V{U})^{-1}\V{U}^H\\
\Longleftrightarrow &\V{U}(\V{M}+\V{U}^H\V{U})^{-1}\V{U}^H=\V{I}-\SB{\V{I}+\V{U}\V{M}^{-1}\V{U}^H}^{-1}\\
\Longrightarrow &\Nm{\V{U}(\V{M}+\V{U}^H\V{U})^{-1}\V{U}^H}=\Nm{\V{I}-\SB{\V{I}+\V{U}\V{M}^{-1}\V{U}^H}^{-1}}\\
\Longrightarrow &\Nm{\V{U}(\V{M}+\V{U}^H\V{U})^{-1}\V{U}^H}=\frac{\Nm{\V{U}\V{M}^{-1}\V{U}^H}}{1+\Nm{\V{U}\V{M}^{-1}\V{U}^H}}.\\
\end{split}
\end{equation}

\end{proof}
回到推论2.1,(B-11)-(B-17)照旧,(B-18)及之后推导修改:

令$\V{A}_k=\bm{\Sigma}^{-1}_k\V{F}_k\SB{\V{I}+\V{B}_{k-1}^H\bm{\Lambda}^{-1}_{k-1}\V{B}_{k-1}+\V{F}_k^H\bm{\Sigma}^{-1}_k\V{F}_k}^{-\frac{1}{2}}$, 则 $\bm{\Lambda}_k^{-1}=\bm{\Sigma}_k^{-1}-\V{A}_k\V{A}_k^H$.

由Lemma 1得:
\begin{equation}
\begin{split}
&f(k+1)\\
=&\Nm{\V{A}_k^H\V{B}_k\SB{\V{I}+\V{B}_{k}^H\bm{\Lambda}^{-1}_{k}\V{B}_{k}+\V{F}_{k+1}^H\bm{\Sigma}^{-1}_{k+1}\V{F}_{k+1}}^{-1}\V{B}_k^H\V{A}_k}\\
=&\Nm{\V{A}_k^H\V{B}_k\SB{\V{I}+\V{B}_{k}^H\bm{\Sigma}^{-1}_{k}\V{B}_{k}+\V{F}_{k+1}^H\bm{\Sigma}^{-1}_{k+1}\V{F}_{k+1}-\V{B}_{k}^H\V{A}_k\V{A}_k^H\V{B}_{k}}^{-1}\V{B}_k^H\V{A}_k}\\
=&\frac{\Nm{\V{A}_k^H\V{B}_k\SB{\V{I}+\V{B}_{k}^H\bm{\Sigma}^{-1}_{k}\V{B}_{k}+\V{F}_{k+1}^H\bm{\Sigma}^{-1}_{k+1}\V{F}_{k+1}}^{-1}\V{B}_k^H\V{A}_k}}{1-\Nm{\V{A}_k^H\V{B}_k\SB{\V{I}+\V{B}_{k}^H\bm{\Sigma}^{-1}_{k}\V{B}_{k}+\V{F}_{k+1}^H\bm{\Sigma}^{-1}_{k+1}\V{F}_{k+1}}^{-1}\V{B}_k^H\V{A}_k}}\\
=&\frac{g(k)}{1-g(k)}.
\end{split}
\end{equation}

这里,

\begin{equation}
\begin{split}
&g(k)\\
=&\Nm{\V{A}_k^H\V{B}_k\SB{\V{I}+\V{B}_{k}^H\bm{\Sigma}^{-1}_{k}\V{B}_{k}+\V{F}_{k+1}^H\bm{\Sigma}^{-1}_{k+1}\V{F}_{k+1}}\V{B}_k^H\V{A}_k}\\
=&\Nm{\V{A}_k^H \bm{\Sigma}_k^{\frac{1}{2}}\bm{\Sigma}_k^{-\frac{1}{2}} \V{B}_k\SB{\V{I}+\V{B}_{k}^H\bm{\Sigma}^{-1}_{k}\V{B}_{k}+\V{F}_{k+1}^H\bm{\Sigma}^{-1}_{k+1}\V{F}_{k+1}}^{-1}\V{B}_k^H \bm{\Sigma}_k^{-\frac{1}{2}}\bm{\Sigma}_k^{\frac{1}{2}} \V{A}_k}\\
\leq & \Nm{\V{A}_k^H \bm{\Sigma}_k^{\frac{1}{2}}\bm{\Sigma}_k^{-\frac{1}{2}} \V{A}_k }\Nm{\bm{\Sigma}_k^{-\frac{1}{2}} \V{B}_k\SB{\V{I}+\V{B}_{k}^H\bm{\Sigma}^{-1}_{k}\V{B}_{k}+\V{F}_{k+1}^H\bm{\Sigma}^{-1}_{k+1}\V{F}_{k+1}}^{-1}\V{B}_k^H \bm{\Sigma}_k^{-\frac{1}{2}}}\\
=& \Nm{\bm{\Sigma}_k^{\frac{1}{2}} \V{A}_k \V{A}_k^H \bm{\Sigma}_k^{\frac{1}{2}}}\Nm{\bm{\Sigma}_k^{-\frac{1}{2}} \V{B}_k\SB{\V{I}+\V{B}_{k}^H\bm{\Sigma}^{-1}_{k}\V{B}_{k}+\V{F}_{k+1}^H\bm{\Sigma}^{-1}_{k+1}\V{F}_{k+1}}^{-1}\V{B}_k^H \bm{\Sigma}_k^{-\frac{1}{2}}}\\
=& \Nm{\bm{\Sigma}_k^{-\frac{1}{2}} \V{F}_k\SB{\V{I}+\V{B}_{k-1}^H\bm{\Lambda}^{-1}_{k-1}\V{B}_{k-1}+\V{F}_{k}^H\bm{\Sigma}^{-1}_{k}\V{F}_{k}}^{-1}\V{F}_k^H \bm{\Sigma}_k^{-\frac{1}{2}}}\\
 &\cdot\Nm{\bm{\Sigma}_k^{-\frac{1}{2}} \V{B}_k\SB{\V{I}+\V{B}_{k}^H\bm{\Sigma}^{-1}_{k}\V{B}_{k}+\V{F}_{k+1}^H\bm{\Sigma}^{-1}_{k+1}\V{F}_{k+1}}^{-1}\V{B}_k^H \bm{\Sigma}_k^{-\frac{1}{2}}}\\
 =&\frac{\Nm{\bm{\Sigma}_k^{-\frac{1}{2}} \V{F}_k\SB{\V{I}+\V{B}_{k-1}^H\bm{\Lambda}^{-1}_{k-1}\V{B}_{k-1}}^{-1}\V{F}_k^H \bm{\Sigma}_k^{-\frac{1}{2}}}}{1+\Nm{\bm{\Sigma}_k^{-\frac{1}{2}} \V{F}_k\SB{\V{I}+\V{B}_{k-1}^H\bm{\Lambda}^{-1}_{k-1}\V{B}_{k-1}}^{-1}\V{F}_k^H \bm{\Sigma}_k^{-\frac{1}{2}}}}\\
 &\cdot\frac{\Nm{\bm{\Sigma}_k^{-\frac{1}{2}} \V{B}_k\SB{\V{I}+\V{F}_{k+1}^H\bm{\Sigma}^{-1}_{k+1}\V{F}_{k+1}}^{-1}\V{B}_k^H \bm{\Sigma}_k^{-\frac{1}{2}}}}{1+\Nm{\bm{\Sigma}_k^{-\frac{1}{2}} \V{B}_k\SB{\V{I}+\V{F}_{k+1}^H\bm{\Sigma}^{-1}_{k+1}\V{F}_{k+1}}^{-1}\V{B}_k^H \bm{\Sigma}_k^{-\frac{1}{2}}}}\\
 \leq & \SB{\frac{c}{c+1}}^2<\frac{1}{2}.
\end{split}
\end{equation}

故而

\begin{equation}
f(k+1)=\frac{g(k)}{1-g(k)}\leq \frac{c^2}{2c+1}<1.
\end{equation}

\end{document}