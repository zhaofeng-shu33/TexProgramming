\documentclass{article}% a book has chapters but an article doesn't
\usepackage{bookmark}
\usepackage{ifthen}
\usepackage{xparse}
\usepackage{enumitem}
\usepackage{xeCJK}
\usepackage{bm}
\begin{document}
\makeatletter
\newcommand{\thu@denotation@name}{Main Symbol Table}
\newcounter{thu@bookmark}
\NewDocumentCommand\thu@chapter{s o m o}{
  \IfBooleanF{#1}{%
    \ClassError{thuthesis}{You have to use the star form: \string\thu@chapter*}{}
  }%
  %\if@openright\cleardoublepage\else\clearpage\fi\phantomsection%
  \IfValueTF{#2}{%
    \ifthenelse{\equal{#2}{}}{%
      \addtocounter{thu@bookmark}\@ne
      \pdfbookmark[0]{#3}{thuchapter.\thethu@bookmark}
    }{%
      \addcontentsline{toc}{chapter}{#3}
    }
  }{%
    \addcontentsline{toc}{chapter}{#3}
  }%
  \section*{#3}%
  \IfValueTF{#4}{%
    \ifthenelse{\equal{#4}{}}
    {\@mkboth{}{}}
    {\@mkboth{#4}{#4}}
  }{%
    \@mkboth{#3}{#3}
  }
}

\newlist{thu@denotation}{description}{1}
\setlist[thu@denotation]
{%
  nosep,
  font=\normalfont,
  align=left,
  leftmargin=!, % sum of the following 3 lengths
  labelindent=0pt,
  labelwidth=2.5cm,
  labelsep*=0.5cm,
  itemindent=0pt,
}
\newenvironment{denotation}[1][2.5cm]
{
  \thu@chapter*[]{\thu@denotation@name} % no tocline
  \vskip-30bp
\begin{thu@denotation}[labelwidth=#1]
}
{
  \end{thu@denotation}
}
\makeatother

\begin{denotation}[3cm]
\item[$N_b$] 锚点数量
\item[$\bm{A}^{\mathrm{T}}$] 矩阵$\bm{A}$的转置
\item[$\bm{p}_i$] 第i个待定位节点的位置,在时间协作中表示目标节点各个时刻的位置
\item[$\bm{P}$] 全部要估计的待定位节点的位置向量,即$(\bm{p}_1,\bm{p}_2,\dots,\bm{p}_{N_a})$
\item[$\bm{p}^b_i$] 第i个锚点的位置
\item[$f(x_1,\dots,x_n|\theta)$] 参数$\theta$已知条件下随机变量$X_1,\dots,X_n$联合概率密度函数
\item[$\bm{J}(\bm{p})$] 关于待定位节点位置$\bm{p}$的费舍尔信息矩阵
\item[$N_a$] 待定位节点数量
\item[$\bm{u}_{i,j}$] 节点i和节点j之间的单位方向向量,时间协作中表示采样时刻$t_i$和$t_j$之间目标节点的位置变化向量
\item[$\bm{C}_{i,j}$] 费舍尔信息矩阵位于i行j列的元素的相反数
\item[$\lambda_{i,j}$] 节点$i,j$之间的 RII
\item[$\bm{u}_{i,j}$] 节点$i,j$之间的单位方向向量,时间协作中表示某节点第$i$个时刻的位置和第$j(j=i+1)$个时刻的位置之间的单位方向向量
\item[$\phi_{i,j}$] $\bm{u}_{i,j}$的方向角
\item[$\bm{A}^{-1}_{1\times2,1\times2}$] 方阵$\bm{A}$的逆阵的取前两行和前两列决定的子矩阵
\end{denotation}


\end{document}