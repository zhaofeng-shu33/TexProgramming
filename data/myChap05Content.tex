\chapter{结论}\label{cha:content5}
    在本文中,我们从协作定位的数学模型出发,通过对费舍尔信息矩阵的分析,研究了定位网络中信息时空传播的机理。首先,我们对费舍尔信息矩阵进行了预处理,将其归为某种类型的矩阵加以研究;同时由于很多时候直接推导存在很大的困难,我们采用猜想-归纳-证明的思路,用数学归纳法给出了部分结论的严谨的证明;
    除此之外,针对推导得出比较复杂的表达式,我们采用了瑞利商、黎曼积分以及连分式等思路进行化简。
    最后,我们结合数值仿真对结论进行了必要的验证。
  \section{已取得的成果}
  % Keep the summary *very short*.
  在数学方法方面,本文主要的成果如下:
  \begin{itemize}
  \item
    使用复数表示法推导得出非协作定位场景下费舍尔信息矩阵的特征值和特征向量的表达式。
  \item
    推导得出秩一矩阵的克罗内克积对N维对称正定矩阵扰动后行列式的表达式。
  \item
    推导得出二维场景下特殊完全图的邻接矩阵所有特征值,其中使用瑞利商给出了最大 特征值的表达式。
  \item 推导得出二维场景下特殊度为2的图的邻接矩阵的所有特征值;当网络规模趋向无穷大时,求出了所有特征值的倒数和的平均值的极限。
  \item 使用连分式推导得出形如式(\ref{eq:Pab})的对称正定矩阵$\bm{A}$确定的$\bm{A}^{-1}_{1\times2,1\times2}$的两个特征值;分析得出了决定特征值的连分式的序列指数收敛的特性,并做出适当的推广。
  \end{itemize}
  \section{工作中的不足之处}
  \begin{itemize}
  \item
  关于两个节点协作的场景中的最优部署角度的结论只是充分性条件,而且结论本身缺少工程直观。
  \item
  考虑的全连接网络过于特殊,而且全连接网络在工程实际中也不现实。
  \item
  关于大规模正方形和正六边形网络的讨论直接给出了结论,由于过程相对很难表述而缺少必要的数学推导或证明。
  \end{itemize}
  \section{未来展望}

  数学方面,本文在线性网络中没有处理成环的情形,环状线性网络对应的费舍尔信息矩阵是循环三对角块状矩阵,应该也可以做出一些好的结果。
  工程方面,本文着重于对网络中信息耦合机理与网络中角度这个几何参量的分析,所研究的模型比较简单,可能与实际问题有一定的出入。今后的工作可以结合计算机仿真工具对复杂网络拓扑下的定位误差作深入的探讨。

