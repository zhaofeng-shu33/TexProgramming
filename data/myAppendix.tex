%\documentclass[type=bachelor]{thuthesis}
%\documentclass{article}
%\usepackage{xeCJK}%preamble part
%\usepackage{graphicx}
%\usepackage{indentfirst}
%\usepackage{enumerate}
%\usepackage[a4paper, inner=1.5cm, outer=3cm, top=2cm, bottom=3cm, bindingoffset=1cm]{geometry}
%\usepackage{epstopdf}
%\usepackage{listings}
%\usepackage{array}
%\usepackage{fontspec}
%\usepackage{bm}
%\usepackage{gensymb}
%\usepackage{todonotes}
%\usepackage{amsmath, amstheorem, amssymb}
%\usepackage[citecolor=blue]{hyperref}
%\newtheorem{definition}{Definition}
%%\newtheorem{theorem}{Theorem}[section]
%\newtheorem{theorem}{Theorem}
%\newtheorem{corollary}{Corollary}
%\newtheorem{lem}{Lemma}
%\newtheorem{remark}{Remark}
%\DeclareMathOperator{\sgn}{sgn}
%\theoremstyle{remark}
%\newtheorem*{rem}{Remark}
%\setCJKmainfont[BoldFont={SimHei}]{SimSun}
%\setCJKmonofont{SimSun}
%\setmainfont{Times New Roman}
%\newCJKfontfamily[hei]\heiti{SimHei}
%\setlength{\extrarowheight}{4pt}
%\setlength{\parindent}{1cm}
%
%\pagenumbering{arabic} 
%\begin{document}
%\setcounter{page}{40}
%\title{协作定位中的信息耦合} 
%\author{\fontsize{12pt}{\baselineskip}{沈渊,清华大学电子工程系副教授}}
%\maketitle
%\begin{appendix}
\chapter{外文资料的调研阅读报告或书面翻译}
\title{第一篇--协作定位中的信息耦合$^{[1]}$}
\textbf{摘要}:
%第一篇论文翻译,原文的题目是
%Information Coupling in Cooperative Localization
%,原文的摘要翻译如下:
不依赖环境的高精度协作定位网络能有一系列的重要的应用。但是现有的分布式协作定位算法没有考虑到预测节点位置彼此之间的相关性。这篇文章通过费舍尔信息量的度量研究了协作定位网络的相关性问题,产生了信息耦合这个概念。为了描述这个特性,我们重点关注了最简单的非平凡情形并且推导出了信息耦合的表达式。
\section{符号约定}
$\text{tr}{\bm{A}},\text{adj}{\bm{A}}$和$
|\bm{A}|$分布表示方阵A的迹,伴随矩阵和行列式。
$[\cdot]^T$表示变量的转置;$\mathbb{S}^2,\mathbb{S}^2_+,\mathbb{S}^2_{++}$分别表示$2\times 2$的实矩阵、半正定矩阵和正定矩阵。另外$\angle \{ \bm{u},\bm{v}\}
$表示向量$\bm{u}$和向量$\bm{v}$之间的夹角。
\section{协作定位中的联合估计}
考虑一个有$\mathcal{N}_a$个移动节点和$\mathcal{N}_b$个移动节点的协作定位网络,锚点的位置已知${\bm{p}_j:j \in \mathcal{N}_b}$,并且移动节点尝试通过和邻居节点的测距和通信确定它们自己的位置${\bm{p}_k: k \in \mathcal{N}_a}$。
在文献[2]确定的测距信息是FIM的基本组成模块,这种测距信息描述了关于从测量中获得的距离信息的强度和方向。
\begin{definition}
在节点k和j之间的总的测距信息强度(RII)定义为关于从它们之间的距离测量中得到的距离$d_{k,j}=||\bm{p}_k-\bm{p}_j||$的费舍尔信息量。
\end{definition}
\begin{definition}
设$\bm{u}_{i_1,i_2}$为从节点$i_1$到节点$i_2$的单位方向向量。定义$\bm{C}^{n,m}_{k,j}$是关于$k,j,n,m \in \mathcal{N}_a \cup \mathcal{N}_b$矩阵:
\[
\bm{C}^{n,m}_{k,j} \triangleq \frac{\bm{u}_{k,j}\bm{u}_{n,m}^T+\bm{u}_{n,m}\bm{u}_{k,j}^T}{2} \in \mathbb{S}^2
\]
另外,定义
\[
\mathring{\bm{C}}^{n,m}_{k,j} \triangleq \frac{\mathring{\bm{u}}_{k,j}\mathring{\bm{u}}_{n,m}^T+\mathring{\bm{u}}_{n,m}\mathring{\bm{u}}_{k,j}^T}{2} \in \mathbb{S}^2
\]
\end{definition}

其中$\mathring{\bm{u}}_{i_1,i_2}$为逆时针方向上垂直于$\bm{u}_{i_1,i_2}$的单位向量,另外,为记号上的简便,$\bm{C}_{k,j}=\bm{C}^{k,j}_{k,j} \in \mathbb{S}^2_+$。
EFIM的概念[6]让我们能直接通过Schur补的方法从FIM中推导对于参数向量的一个子集的信息不等式。由于在已知移动节点位置的条件下测距彼此之间相互独立,对于移动节点位置的EFIM可以写成闭式解的形式。用求导的链式法则可以证明,对于协作定位这种EFIM是下面分块矩阵的形式:
\begin{equation}\label{eq:FIM}
\bm{J}_e=\left[\begin{array}{cccc}
\bm{K}_1^{\mathcal{N}_a \backslash \{1\}} & -\zeta_{1,2} \bm{C}_{1,2} &\cdots & -\zeta_{1,N_a} \bm{C}_{1,N_a} \\
-\zeta_{1,2} \bm{C}_{1,2} &\bm{K}_2^{\mathcal{N}_a \backslash \{2\}} & \cdots & -\zeta_{2,N_a} \bm{C}_{2,N_a} \\
\vdots & \vdots & \ddots & \vdots\\
-\zeta_{1,N_a} \bm{C}_{1,N_a} & -\zeta_{2,N_a} \bm{C}_{2,N_a} & \cdots & \bm{K}_2^{\mathcal{N}_a \backslash \{N_a\}}\\
\end{array}
\right]
\end{equation}
其中$\mathcal{N}_a=\{1,2 \dotsc N_a\}$,对于$k,j \in \mathcal{N}_a,\zeta_{k,j}$是节点k和节点j总的RII,并且
\[
\bm{K}^{\mathcal{N}}_k=\bm{J}^{\mathcal{N}_b}_k+\sum_{j\in \mathcal{N}} \zeta_{k,j} \bm{C}_{k,j}, \mathcal{N} \subset \mathcal{N}_a
\]
上式中$\bm{J}^{\mathcal{N}_b}_k$表示仅从和$\mathcal{N}_b$测距中获得的关于第k个节点的EFIM。
\begin{remark}
对于式(\ref{eq:FIM})中表示协作定位的EFIM不是对角矩阵,反映了从达到CRLB的位置估计量推断中的移动节点间位置信息是相关的。这种情况阻碍了针对协作定位的最优或次优的分布式算法的设计。%development->设计?
因此在接下来的分析中我们会探究由于非对角结构引起的信息耦合的表现。
\end{remark}
\section{信息耦合}
为获得信息耦合的洞见,我们考虑一个含有$\mathcal{N}_b$个移动节点和三个协作节点的网络:$\mathcal{N}_a=\{1,2,3\}$,这代表了一个最简单的非平凡的信息耦合的情形。下面我们推导每一个移动节点的EFIM和它的逆的闭式解。
\begin{definition}
给定$\zeta_{k,j} \in (0,\infty)$和$\bm{J} \in \mathbb{S}^2_{++}$。定义$\Phi_{k,j}(\bm{J})$是如下形式的商:
\[
\Phi_{k,j}(\bm{J})\triangleq=\frac{|\bm{J}|}{|\bm{J}+\zeta_{k,j}\bm{C}_{k,j}|}\in (0,1).
\]
\end{definition}
\begin{remark}
注意到$\forall \bm{J} \in \mathcal{S}^2_{++},\lim_{\zeta_{k,j} \rightarrow 0}\Phi_{k,j}=1$并且$\lim_{\zeta_{k,j} \rightarrow \infty}\Phi_{k,j}=0$.此外,$\forall \zeta_{k,j} \in (0,\infty)$,$\lim_{|\bm{J}|\rightarrow 0}\Phi_{k,j}=0$并且$\lim_{|\bm{J}|\rightarrow \infty}\Phi_{k,j}=1.$又因为$|\bm{J}+\zeta_{k,j}|=|\bm{J}|+\zeta_{k,j}\bm{u}_{k,j}\text{adj}\{\bm{J}\}\bm{u}_{k,j},$
\[
\frac{\mu_2}{\mu_2+\zeta_{k,j}}\leq \Phi_{k,j}(\bm{J})\leq \frac{\mu_1}{\mu_1+\zeta_{k,j}}
\]
其中 $\mu_1 \geq \mu_2 \ge 0$是$\bm{J}$的两个特征根。$\Phi_{k,j}(\bm{J})$表示从节点j获得的$\zeta_{k,j}$(RII)中可以被节点k有效利用的部分,而$\bm{J}$是这个部分中的不确定性。
\end{remark}
\begin{theorem}
设$\bm{J}_1^{\mathcal{N}_b},\bm{J}_2^{\mathcal{N}_b},\bm{J}_3^{\mathcal{N}_b}$分别表示仅从锚点$\mathcal{N}_b$获得的EFIM,则移动节点1的EFIM由下式给出:
\begin{equation}
\bm{J}_1=\bm{J}_1^{\mathcal{N}_b}+\check{\zeta}_{1,2}\bm{C}_{1,2}+\check{\zeta}_{1,3}\bm{C}_{1,3}+\kappa_{2,3}\bm{C}^{1,3}_{1,2}
\end{equation}
其中
\begin{eqnarray}
\check{\zeta}_{1,2}&=\zeta_{1,2} \cdot \Phi_{1,2}(\bm{J}_2^{\mathcal{N}_b}+\zeta_{2,3} \cdot \Phi_{2,3}(\bm{K}_3^{1})\cdot \bm{C}_{2,3})\\
\check{\zeta}_{1,3}&=\zeta_{1,3} \cdot \Phi_{1,3}(\bm{J}_3^{\mathcal{N}_b}+\zeta_{2,3} \cdot \Phi_{2,3}(\bm{K}_2^{1})\cdot \bm{C}_{2,3})
\end{eqnarray}
并且$\kappa_{2,3}$由(5)式给出。
此外,EFIM的逆为:
\begin{equation}
\bm{J}^{-1}=\frac{1}{|\bm{J}_1|}[\text{adj}\{\bm{J}^{\mathcal{N}_b}_1\}+\check{\zeta}_{1,2}\mathring{\bm{C}}_{1,2}+\check{\zeta}_{1,3}\mathring{\bm{C}}_{1,3}+\kappa_{2,3}\mathring{\bm{C}}_{1,2}^{1,3}]
\end{equation}
其中$|\bm{J}_1|$由(7)式给出。
\end{theorem}
\begin{remark}
移动节点1的EFIM和它的逆都是三项的和,分别对应着从锚点、协作获取的信息以及耦合项。特别的,在(2)式中,第一项$\bm{J}_1^{\mathcal{N}_b}$是从锚点获取的信息,第二项$\check{\zeta}_{1,2}\bm{C}_{1,2}+\check{\zeta}_{1,3}\bm{C}_{1,3}$是从节点12连线和节点13连线获得的信息增量,这个增量取决于RII和由(3)式和(4)式给出的协作节点位置的不确定性。第三项$\kappa_{2,3}\bm{C}_{1,2}^{1,3}$是来自节点2和3的信息耦合项,这一项的的出现是由于节点2和3彼此之间也有协作。在描述EFIM的逆的(6)式中,三项共同的伸缩因子是行列式$|\bm{J}_1|$的倒数,并且后面的每个矩阵都是由原来的单位向量逆时针转90度再做外积得到的。
\end{remark}
从(5)式可以得到,$\kappa_{2,3}$的一个上界是:
\[
\kappa_{2,3}\leq 2\zeta_{1,2}\zeta_{1,3}\zeta_{2,3}|\bm{u}_{1,2}^T(\bm{J}_2^{\mathcal{N}_b})^{-1}\bm{C}_{2,3}(\bm{J}_3^{\mathcal{N}_b})^{-1}\bm{u}_{1,3}|
\]
从上式可以看出,在如下情形中没有耦合:
\begin{enumerate}[(i)]
\item{如果$\angle \{\bm{u}_{1,2},\bm{J}_2^{\mathcal{N}_b}\bm{u}_{2,3}\}$或者$\angle \{\bm{u}_{1,3},\bm{J}_3^{\mathcal{N}_b}\bm{u}_{2,3}\}$为90度,那么$\kappa_{2,3}=0$;}
\item{如果移动节点2成为一个锚点,也就是$|\bm{J}_2^{\mathcal{N}_b}|$趋向于无穷大,那么$\check{\zeta_{1,2}}=\zeta_{1,2},\check{\zeta_{1,3}}=\zeta_{1,3}\cdot \Phi_{1,3}(\bm{J}_3^{\mathcal{N}_b}+\zeta_{2,3}\bm{C}_{2,3}),$并且$\kappa_{2,3}=0$;}
\item{如果节点2和节点3之间不协作,也就是$\zeta_{2,3}=0$,那么$\check{\zeta_{1,2}}=\zeta_{1,2}\cdot \Phi_{1,2}(\bm{J}_2^{\mathcal{N}_b},\check{\zeta_{1,3}}=\zeta_{1,3}\cdot \Phi_{1,3}(\bm{J}_3^{\mathcal{N}_b}),$并且$\kappa_{2,3}=0$;}
\end{enumerate}
这些结果说明了当移动节点的位置满足(i)中的正交性条件或者如$(ii)$或$(iii)$给出的有少于3个节点参与协作时,不会有耦合项出现。
\begin{corollary}
耦合项$\kappa_{2,3}\bm{C}_{1,2}^{1,3}$有特征值$[\cos(\angle \{\bm{u}_{1,2},\bm{u}_{1,3}\})+1]\kappa_{2,3}/2$和$[\cos(\angle \{\bm{u}_{1,2},\bm{u}_{1,3}\})-1]\kappa_{2,3}/2$,分别对应的特征向量是$
\bm{u}_{1,2}+\bm{u}_{1,3}$和$\bm{u}_{1,2}-\bm{u}_{1,3}$。
\end{corollary}
\begin{remark}
这个引理说明了如果$\kappa_{2,3}>0$(对应的,如果$\kappa_{2,3}<0$),如果忽略耦合项,由协作获得的信息椭圆在$
\bm{u}_{1,2}+\bm{u}_{1,3}$方向上会被低估(对应的,高估),而在$
\bm{u}_{1,2}-\bm{u}_{1,3}$方向上会被高估(对应的,低估)。另外当角$\angle \{\bm{u}_{1,2},\bm{u}_{1,3}\}$是锐角(对应的,钝角)时,在$
\bm{u}_{1,2}+\bm{u}_{1,3}$方向上的信息耦合会比$
\bm{u}_{1,2}-\bm{u}_{1,3}$方向上的更显著(对应的,不如前者显著)。
\end{remark}
\begin{figure}
\centering
\includegraphics[width=\textwidth/2]{figure_2.png}
\caption{协作定位中的信息耦合可能会严重实际可达到的定位信息}
\end{figure}

举例说明如下,图1描述了考虑和忽略耦合项的信息椭圆的形状。在这两种类别中,从锚点获得的EFIM为简便取成$\bm{J}^{\mathcal{N}_b}_1,\bm{J}^{\mathcal{N}_b}_2,\bm{J}^{\mathcal{N}_b}_3=\text{diag}\{1,1\}$,并且三个协作节点分别位于位置$\bm{p}_1,\bm{p}_2,\bm{p}_3$.
图1展示了节点1的真实信息椭圆$\bm{J}_1$和忽略耦合项后的信息椭圆$\bm{\tilde{J}}_1$的差别。
%\newpage
%% \documentclass[10pt,journal,compsoc]{IEEEtran}
% \usepackage{xeCJK}%preamble part
% \usepackage{graphicx}
% \usepackage{pdfpages}
% \usepackage{indentfirst}
% \usepackage{enumerate}
% \usepackage[a4paper, inner=1.5cm, outer=3cm, top=2cm, bottom=3cm, bindingoffset=1cm]{geometry}
% \usepackage{epstopdf}
% \usepackage{listings}
% \usepackage{array}
% \usepackage{fontspec}
% \usepackage{bm}
% \usepackage{gensymb}
% \usepackage{todonotes}
% \usepackage{amsmath, amsthm, amssymb}
% \usepackage[citecolor=blue]{hyperref}
% \newtheorem{definition}{Definition}
%\newtheorem{thm}{Theorem}[section]
% \newtheorem{thm}{Theorem}
% \newtheorem{corollary}{Corollary}
% \newtheorem{lem}{Lemma}
% \newtheorem{remark}{Remark}
% \newtheorem{proposition}{Proposition}
% \DeclareMathOperator{\sgn}{sgn}
% \theoremstyle{remark}
% \newtheorem*{rem}{Remark}
% \setCJKmainfont[BoldFont={SimHei}]{SimSun}
% \setCJKmonofont{SimSun}
% \setmainfont{Times New Roman}
% \newCJKfontfamily[hei]\heiti{SimHei}
% \setlength{\extrarowheight}{4pt}
% \setlength{\parindent}{1cm}
 
%\begin{document}
\title{第二篇--协作定位网络中的时空信息耦合$^{[2]}$} 
%\author{\fontsize{12pt}{\baselineskip}{沈渊,清华大学电子工程系副教授}}
%\markboth{Journal of \LaTeX\ Class Files,~Vol.~14, No.~8, August~2015}%
%{Shell \MakeLowercase{\textit{et al.}}: Bare Advanced Demo of IEEEtran.cls for IEEE Computer Society Journals}
%\IEEEtitleabstractindextext{
%\begin{abstract}
\textbf{摘要}:%第二篇论文翻译,原文的题目是
%Spatio-Temporal Information Coupling in Cooperative Network Navigation
%原文的摘要翻译如下:
可靠的定位信息对很多基于位置的应用起着至关重要的作用。通过时空联合协作的网络导航可以给移动节点提供高精度和鲁棒的位置信息。同时,由于待测节点的位置相关,这种联合协作导致了错综复杂的信息获取方式,也就是信息耦合的问题。在这篇文章中,通过对费舍尔信息量的分析,我们在四种有代表性的情形中量化了信息耦合。我们说明了每个节点所获得的信息来自与它进行时空协作的节点和由于和邻居节点的协作产生的信息耦合。我们的结果为网络中复杂信息获取提供了洞见,并且能够为高效的网络导航算法提供指导。
%\end{abstract}
%}\maketitle
%reset section counter here
\setcounter{section}{0}
\section{简介}
关于研究背景的翻译略去,下面是原文一些记号上的约定。
$[\cdot]^T$表示变量的转置;$\mathbb{S}^D,\mathbb{S}^D_+,\mathbb{S}^D_{++}$分别表示$D\times D$的实矩阵、半正定矩阵和正定矩阵。$\bm{J}_r(\bm{v}):=\bm{v}\bm{v}^T$表示由向量$\bm{v}$做外积得到的秩1阵。
另外$\angle \{ \bm{u},\bm{v}\}
$表示向量$\bm{u}$和向量$\bm{v}$之间的夹角。
\section{网络导航中的费舍尔信息量}
在本节中,我们首先介绍网络模型和网络导航中的FIM作为预备知识,然后描述这篇文章重点讨论的4种场景。
\subsection{预备知识}
考虑一个由若干节点构成的协作网络,用$\bm{x}_k^{(n)}\in \mathbb{R}^D$表示节点k在时间$t_n$的位置状态,$k=1,2,....,N$且$n=1,2,...T$.网络导航的目标是从测量和先验信息中推断节点的位置信息。
令$\mathcal{S}=\{1,2,...,S\}\text{ with }S=N\cdot T$是位置状态的下标集,文献[8]已经证明了对于S个位置状态FIM可以分解成:
\begin{equation}
\bm{J}=\sum_{(i,j)\in \mathcal{S}^2,i\geq j} \bm{G}^S_{i,j} \otimes \bm{K}_{i,j}
\end{equation}
其中$\otimes$表示Kronecker矩阵积,$S\times S$的矩阵$\bm{G}^S_{i,j}$的元素$(k,r)$由下式给出:
\[
[\bm{G}^S_{i,j}]_{k,r}=\left\{
\begin{array}{c l}	
     1 & k=r=i\\
     1 & k=r=j\\
     -1 & k=i,r=j,k \neq r\\
     -1 & k=j,r=i,k \neq r\\
     0, & \text{otherwise}
\end{array}\right.
\]
$\bm{K}_{i,j}\in \mathbb{S}_+^D$ 描述了包含在测量或先验知识中和位置状态i,j有关的位置信息。$\bm{K}_{i,j}=\bm{K}_{j,i}$并且在缺少关于位置状态i,j的测量或先验知识的情况下$\bm{K}_{i,j}=0$。特别的,如果$i=j$,$\bm{K}_{i,i}$描述了仅仅和位置状态i有关的位置信息。
\subsection{有代表性的场景}
我们考虑图1中的四种场景以获得对时空协作的洞见。
\begin{figure}
\centering
\includegraphics[width=\textwidth/2]{figure_1.png}
\caption{四种场景:(A)两个节点的空间协作(红色的连接);(B)三个节点的空间协作;(C)一个节点的时间协作(绿色的连接);(D)两个节点的时空协作。}
\end{figure}

在这些场景中,位置状态就是节点的位置,节点通过和位置已知的锚点和位置未知的邻居节点的测距中获得位置信息。
在接下来的讨论中,在时间步为n时,$\bm{K}^{(n)}_{i,i}$表示节点i从锚点获取的信息;$\bm{u}_{i,j}^{(n)}$表示节点i和j连线的单位方向向量;$\lambda^{(n)}_{i,j}=\lambda^{(n)}_{j,i}$表示节点i和j之间的测距信息强度(RII);$\bm{u}_{i,i}^{(n)}$表示链接节点i在时间步为n和n+1时的位置的单位方向向量;$\lambda^{(n)}_{i,i}$表示节点i的速度信息强度(SII)。

\textbf{场景A:}两个节点通过彼此测距协作来确定它们的位置,即位置状态是$\bm{p}_1$和$\bm{p}_2$,测距信息是$\bm{J}_r(\bm{v}_{1,2})$,其中$\bm{v}_{1,2}=\sqrt{\lambda_{1,2}}\bm{u}_{1,2}$,因此,基于(1)式我们得到这种场景的FIM是:
\begin{equation}
\bm{J}_A=\left[
\begin{array}{cc}
\bm{K}_{1,1}+\bm{J}_r(\bm{v}_{1,2})&-\bm{J}_r(\bm{v}_{1,2})\\
-\bm{J}_r(\bm{v}_{1,2})&\bm{K}_{2,2}+\bm{J}_r(\bm{v}_{1,2})\\
\end{array}
\right]
\end{equation}
%\begin{figure*}[!t]
\begin{equation*}
\bm{J}_B=\left[
\begin{array}{ccc}
\bm{K}_{1,1}+\bm{J}_r(\bm{v}_{1,2})+\bm{J}_r(\bm{v}_{1,2}) & -\bm{J}_r(\bm{v}_{1,2}) & -\bm{J}_r(\bm{v}_{1,3})\\
-\bm{J}_r(\bm{v}_{1,2}) &\bm{K}_{2,2}+\bm{J}_r(\bm{v}_{1,2})+\bm{J}_r(\bm{v}_{2,3}) & -\bm{J}_r(\bm{v}_{2,3})\\
 -\bm{J}_r(\bm{v}_{1,3}) & -\bm{J}_r(\bm{v}_{2,3})&
\bm{K}_{3,3}+\bm{J}_r(\bm{v}_{1,3})+\bm{J}_r(\bm{v}_{2,3})\\
\end{array}
\right]
\end{equation*}
%\hrulefill
%\vspace*{4pt}
%\end{figure*}

\textbf{场景B:} 三个节点通过彼此测距协作确定它们的位置,即位置状态是$\bm{p}_1,\bm{p}_2$和$\bm{p}_3$,测距信息是$\bm{J}_r(\bm{v}_{i,j})$对于$i,j\in \{1,2,3\},i\neq j$,其中$\bm{v}_{i,j}=\sqrt{\lambda_{i,j}}\bm{u}_{i,j}$,因此,基于式(1)我们可以得到这种情形的FIM由(3)给出,在下一页的最开始的地方。

\textbf{场景C:}单节点在两个不同的时刻通过速度测量协作来确定自身的位置,即位置状态是$\bm{p}^{(1)}_1$和$\bm{p}^{(2)}_1$,速度测量的信息是$\bm{J}_r(\bm{t})$,其中$\bm{t}=\sqrt{\lambda^{(1)}_{1,1}}\bm{u}_{1,1}^{(1)}$.因此基于(1),我们得到这种场景的FIM是:
\[\bm{J}_C=\left[
\begin{array}{cc}
\bm{K}_{1,1}^{(1)}+\bm{J}_r(\bm{t})&-\bm{J}_r(\bm{t})\\
-\bm{J}_r(\bm{t})&\bm{K}_{2,2}^{(2)}+\bm{J}_r(\bm{t})\\
\end{array}\right]
\]

\textbf{场景D:}两个节点在两个不同的时刻协作来确定它们的位置,即,位置状态是$\bm{p}_1^{(1)},\bm{p}_2^{(1)},\bm{p}_1^{(2)}$和$\bm{p}_2^{(2)}$,每个节点测量自身的速度和相对另一节点的距离,并且
\begin{itemize}  
\item 测距信息是$\bm{J}_r(\bm{v}_{1,2})$和$\bm{J}_r(\bm{w}_{i,j})$,其中$\bm{v}_{1,2}=\sqrt{\lambda_{1,2}^{(1)}}\bm{u}^{(2)}_{1,2}$,而$\bm{w}_{1,2}=\sqrt{\lambda_{1,2}^{(2)}}\bm{u}^{(1)}_{1,2}$
\item 测速信息是$\bm{J}_r(\bm{t}_1)$和$\bm{J}_r(\bm{t}_2)$,其中$\bm{t}_i=\sqrt{\lambda_{i,i}^{(1)}}\bm{u}^{(1)}_{i,i}$
\end{itemize}
因此,基于(1)式我们得到这种类形的FIM是
\[
\bm{J}_D=\left[
\begin{array}{cc}
\bm{J}^{(1)}_A+\bm{T}&-\bm{T}\\
-\bm{T}&\bm{J}^{(2)}_A+\bm{T}\\
\end{array}
\right]
\]
其中$\bm{J}^{(1)}_A$和$\bm{J}^{(2)}_A$由(2)式给出,上标代表了时间步长,$\bm{T}=\text{diag}\{\bm{J}_r(\bm{t}_1),\bm{J}_r(\bm{t}_2)\}$.
\section{空间信息耦合}
在这一节中,我们推导了场景A和场景B的EFIM,并且描述了场景B中由于空间协作导致的信息耦合。由于篇幅所限,这篇文章的大部分证明略去。
\begin{proposition}
场景A中节点1的EFI是
\[
\bm{J}_e=\bm{K}_{1,1}+(1-\mu_2^{1,1})\bm{J}_r(\bm{v}_{1,2})
\]
其中
\begin{eqnarray*}
\mu_2^{1,1}=&\bm{v}^T_{1,2}(\bm{K}_{2,2}+\bm{J}_r(\bm{v}_{1,2}))^{-1}\bm{v}_{1,2}\\
=&1-(1+\lambda_{1,2}\bm{u}^T_{1,2}\bm{K}^{-1}_{2,2}\bm{u}_{1,2})^{-1}.
\end{eqnarray*}
此外,$\bm{J}_e \preceq \bm{K}_{1,1}+\bm{J}_r(\bm{v}_{1,2}).$
\end{proposition}
\begin{remark}
场景A中节点1的EFIM是来自锚点和来自与节点2的协作的信息之和。协作在与节点2的连线方向上增加了信息,而节点1可以有效利用的RII随RII $\lambda_{1,2}$增大而增大而随节点2位置在$\bm{u}_{1,2}$方向的不确定性(即$\bm{u}^T\bm{K}^{-1}_{2,2}\bm{u}_{1,2}$)的增大而减小。此外,有效RII总是小于或等于RII。当锚点在节点1,2连线方向上给节点2提供无穷大的信息量时取等号,而当锚点
在这一方向上不提供任何信息时取0。
\end{remark}
\begin{proposition}
在场景B中节点1的EFIM是
\begin{equation}
\begin{split}
\bm{J}_e=\bm{K}_{1,1}+(1-\mu_2^{1,1})\bm{J}_r(\bm{v}_{1,2})+(1-\mu_3^{1,1})\bm{J}_r(\bm{v}_{1,3})\\
+\delta\bm{J}_r(\mu_2^{1,3}\bm{v}_{1,2}+\mu_3^{1,2}\bm{v}_{1,3})
\end{split}
\end{equation}
其中$\mu_i^{j,k}=\bm{v}^T_{j,i}(\bm{K}_{i,i}+\bm{J}_r(\bm{v}_{1,i}))^{-1}\bm{v}_{k,i}$,并且$\delta=(1+\mu_2^{3,3}+\mu_3^{2,2})^{-1}$.除此之外,$\bm{J}_e \preceq \bm{K}_{1,1}+\bm{J}_r(\bm{v}_{1,2})+\bm{J}_r(\bm{v}_{1,3})$.
\end{proposition}
\begin{remark}
命题2说明了场景B下节点1的EFIM是4项的和。第一项对应着从锚点获取的信息,其他三项对应着由于空间协作获得的信息。第二项和第三项对应着沿着与节点2和节点3的连线方向获取的有效信息,而最后一项代表着由于节点2和3协作造成的信息耦合。最后一项是一个秩一阵,有一个非负的特征值$\delta ||\mu_2^{1,3}\bm{v}_{1,2}+\mu_3^{1,2}\bm{v}_{1,3}||^2$并且这个特征值对应的特征向量是$\mu_2^{1,3}\bm{v}_{1,2}+\mu_3^{1,2}\bm{v}_{1,3}$,这个方向既依赖于RII,又依赖于节点2和3位置的不确定性,还和节点的空间拓扑有关。
\end{remark}
接下来我们将说明节点2和3的协作强度如何影响节点1的EFIM.
\begin{corollary}
设$\lambda_{2,3}$和$\tilde{\lambda}_{2,3}$分别表示节点2和节点3之间两个RII,并且$(\mu_2^{1,3},\mu_3^{1,2},\delta),(\tilde{\mu}_2^{1,3},\tilde{\mu}_3^{1,2},\tilde{\delta})$分别表示命题2中对应的两组参数,而其他参数均相同。那么:
\begin{equation}
\delta \bm{J}_r(\mu_2^{1,3}\bm{v}_{1,2}+\mu_3^{1,2}\bm{v}_{1,3}) \preceq \tilde{\delta} \bm{J}_r(\tilde{\mu}_2^{1,3}\bm{v}_{1,2}+\tilde{\mu}_3^{1,2}\bm{v}_{1,3})
\end{equation}
当且仅当$\lambda_{2,3}\leq \tilde{\lambda}_{2,3}$。此外
\[
\lim_{\lambda_{2,3} \rightarrow \infty}\delta \bm{J}_r(\mu_2^{1,3}\bm{v}_{1,2}+\mu_3^{1,2}\bm{v}_{1,3})=\bar{\delta} \bm{J}_r(\bar{\mu}_2^{1,3}\bm{v}_{1,2}+\bar{\mu}_3^{1,2}\bm{v}_{1,3})
\]
其中
\[
\bar{\mu}_2^{1,3}=\bm{v}_{1,2}^T(\bm{K}_{2,2}+\bm{J}_r(\bm{v}_{1,2}))^{-1}\bm{u}_{2,3}
\]
\[
\bar{\mu}_3^{1,2}=\bm{v}_{1,3}^T(\bm{K}_{3,3}+\bm{J}_r(\bm{v}_{1,3}))^{-1}\bm{u}_{2,3}
\]
\[
\bar{\delta}=(\bm{u}_{2,3}^T((\bm{K}_{2,2}+\bm{J}_r(\bm{v}_{1,2}))^{-1}+(\bm{K}_{3,3}+\bm{J}_r(\bm{v}_{1,3}))^{-1})\bm{u}_{2,3})^{-1}.
\]
\end{corollary}
\begin{remark}
从(5)中可以知道,当节点2,3方向保持不变时,节点1从和节点2和3协作中获得的信息随着节点2和3之间的RII的增大而增大。另外,这个信息量的上界是在节点2和3之间测距是理想情况时取得。
\end{remark}
在下面这个推论中,我们将说明如果从协作中获得的信息是分别和节点2,节点3的测距信息的加权和这种情形。
\begin{corollary}
场景B中节点1的EFIM可以写成
\[
\bm{J}_e=\bm{K}_{1,1}+\eta_1 \bm{J}_r(\bm{v}_{1,2})+\eta_2 \bm{J}_r(\bm{v}_{1,3}),\eta_1,\eta_2 \geq 0
\]
当且仅当至少下面至少有一个条件满足:
\begin{itemize}
\item{三个节点共线}
\item $\lambda_{1,2}\cdot \lambda_{1,3}\cdot \lambda_{2,3}=0$
\item $\bm{u}_{1,2}^T(\bm{K}_{2,2}+\bm{J}_r(\bm(v)_{1,2}))^{-1}\bm{u}_{2,3}=0$
\item $\bm{u}_{1,3}^T(\bm{K}_{3,3}+\bm{J}_r(\bm(v)_{1,3}))^{-1}\bm{u}_{2,3}=0$
\end{itemize}
\end{corollary}
接下来我们将说明如果锚点提供各向同性的信息给节点2,3,那么权重参数$\mu_i^{j,k}$可以写成$\theta_1=\angle \{\bm{u}_{1,2},\bm{u}_{2,3}\}$和$\theta_2=\angle \{\bm{u}_{1,3},\bm{u}_{2,3}\}$的函数。
\begin{corollary}
如果$\bm{K}_{i,i}=\xi_i\bm{I}$对于$i=2,3$成立,则场景B中节点1的EFIM为
\begin{equation*}
\begin{split}
\bm{J}_e=\bm{K}_{1,1}+\frac{\xi_2}{\xi_2+\lambda_{1,2}}\bm{J}_r(\bm{v}_{1,2})+\frac{\xi_3}{\xi_3+\lambda_{1,3}}\bm{J}_r(\bm{v}_{1,3})\\
+\delta\bm{J}_r(\mu_2^{1,3}\bm{v}_{1,2}+\mu_3^{1,2}\bm{v}_{1,3})
\end{split}
\end{equation*}
其中
\[
\mu_2^{1,3}=\frac{\sqrt{\lambda_{1,2}\lambda_{2,3}}\cos \theta_1}{\xi_2+\lambda_{1,2}}
\]
\[
\mu_3^{1,2}=\frac{\sqrt{\lambda_{1,3}\lambda_{2,3}}\cos \theta_2}{\xi_3+\lambda_{1,3}}
\]
\[
\mu_2^{3,3}=\frac{\lambda_{2,3}}{\xi_2}(1-\frac{\lambda_{1,2}\cos^2 (\theta_2)}{\xi_2+\lambda_{1,2}})
\]
\[
\mu_3^{2,2}=\frac{\lambda_{2,3}}{\xi_3}(1-\frac{\lambda_{1,3}\cos^2 (\theta_2)}{\xi_3+\lambda_{1,3}})
\]
\end{corollary}
\begin{remark}
这个结果说明了如果$(i)$节点2和3来自锚点或先验的信息是各向同性的$(ii)\bm{u}_{1,2}\perp\bm{u}_{2,3}$或者$\bm{u}_{1,3}\perp\bm{u}_{2,3}$,那么节点1通过写作获得的信息是分别来自节点2的测距信息的加权和。
\end{remark}
\section{空时信息耦合}
在本节中,我们推导出了场景C和D下节点1在时间步2时的EFIM,并且描述了场景D中由于空时协作产生的信息耦合。
\begin{proposition}
场景C中节点1的EFIM是
\[
\bm{J}_e=\bm{K}_{1,1}^{(2)}+\frac{\lambda_{1,1}^{(1)}}{1+\lambda_{1,1}^{(1)}\bm{u}_{1,1}^T(\bm{K}_{1,1}^{1})^{-1}\bm{u}_{1,1}}\bm{J}_r(\bm{u}_{1,1})
\]
此外$\bm{J}_e \preceq \bm{K}_{1,1}^{(2)}+\bm{J}_r(\bm{t})$.
\end{proposition}
\begin{remark}
与场景A类似,场景C中时间步为2时节点1的EFIM是从锚点获取的信息和与节点1在时间步为1时的协作信息之和。协作在节点1在两个时刻位置连线的方向上增加了信息,而节点1可以有效利用的SII随SII $\lambda_{1,1}^{(1)}$的增加而增大而随着节点1在时间步1的位置不确定性(即方向$\bm{u}_{1,1}^T(\bm{K}_{1,1}^{(1)})^{-1}\bm{u}_{1,1}$)的增大而减小。此外,有效的SII总是小于等于SII,当锚点给时间步1时的节点提供无穷大的信息时取等号,当不提供任何信息时取0.
\end{remark}
\begin{proposition}
场景D中节点1在时间步2时的EFIM是
\begin{equation}
\begin{split}
\bm{J}_e=\bm{K}_{1,1}^{(2)}+&[1-\bm{v}_{1,2}^T(\bm{K}_{2,2}^{(2)}+\bm{H}_{1,2})^{-1}\bm{v}_{1,2}]\bm{J}_r(\bm{v}_{1,2})\\
+&[1-\bm{t}_1^T(\bm{K}_{1,1}^{(1)}+\bm{H}_{1,1})^{-1}\bm{t}_1]\bm{J}_r(\bm{t}_1)\\
+&\delta \bm{J}_r(\nu_2 \bm{v}_{1,2}+\nu_1 \bm{t}_1)
\end{split}
\end{equation}
此外$\bm{J}_e \preceq \bm{K}_{1,1}^{(2)}+\bm{J}_r(\bm{v}_{1,2})+\bm{J}_r(\bm{t}_1)$
\end{proposition}
\begin{remark}
和场景B类似,命题4说明了在场景D中,时间步2时节点1的EFIM是4项的和。第一项对应着从锚点获得的信息,其余的项对应着通过协作获得的信息。第二和第三项分别是节点2在时间步2和节点1进行空间协作以及和自身在时间步1的时间协作获得的信息。而最后一项对应着耦合项。最后一项是一个秩一阵,有一个非负的特征值为$\delta||\nu_2\bm{v}_{1,2}+\nu_1\bm{t}_1||^2$,对应的特征向量为$\nu_2\bm{v}_{1,2}+\nu_1\bm{t}_1$。这个方向依赖于RII,SII,协作节点位置的不确定度以及节点在空间上的拓扑。特别的,当$\bm{t}_2^T(\bm{K}_{2,2}^{(1)})^{-1}\bm{w}_{1,2}=0$时,比如$\bm{K}_{2,2}^{(1)}=\xi_{2,2}^{(1)}\bm{I}$且$\bm{u}_{1,2}^{(1)} \perp\bm{u}_{2,2}^{(1)}$,最后一项为0.
\end{remark}
\section{数值结果}
在本节中,我们给出若干个含有信息耦合的EFIM的数值的例子。特别的,我们将考察节点的网络拓扑如何影响信息耦合项。

信息耦合项是协作节点方向加权组合作外积得到的,对于场景B,权系数是$\sqrt{\delta}\mu_2^{1,3}$和
$\sqrt{\delta}\mu_3^{1,2}$,对于场景D权系数是$\sqrt{\delta}\nu_2$和
$\sqrt{\delta}\nu_1$,图2和图3分别说明了这些系数对于角度$\theta_1=\angle \{\bm{u}_{1,2},\bm{u}_{2,3}\},\theta_2=\angle \{\bm{u}_{1,3},\bm{u}_{2,3}\}$或者$\vartheta_1=\angle \{\bm{u}_{1,1},\bm{u}_{1,2}^{(1)}\},\vartheta_2=\angle \{\bm{u}_{2,2},\bm{u}_{1,2}^{(2)}\}$的依赖关系。在数值结果里,我们设
$\bm{K}_{2,2}=\bm{K}_{3,3}=\bm{K}_{2,2}^{(2)}=\bm{K}_{2,2}^{(1)}=\bm{K}_{1,1}^{(1)}=\bm{I}$,所以的RII和SII都取1,并且$\bm{u}_{1,1}^T\bm{u}_{1,2}^{(2)}=0$.

在图2中,我们可以看到$\sqrt{\delta}\mu_2^{1,3}$和$\sqrt{\delta}\mu_3^{1,2}$,即$\bm{v}_{1,2}$或$\bm{v}_{1,3}$的权重,当$\bm{u}_{1,2}$和$\bm{u}_{1,3}$正交于$\bm{u}_{2,3}$时分别为0.此外,这两项在节点1,2,3共线时达到最大值。

类似的,在图3中,我们可以看到$\sqrt{\delta}\nu_2$和$\sqrt{\delta}\nu_1$,即耦合项中$\bm{v}_{1,2}$或$\bm{t}_1$的系数,当$\bm{u}_{2,2}$正交于$\bm{u}_{1,2}^{(2)}$且$\bm{u}_{1,1}$正交于$\bm{u}_{1,2}^{(1)}$时分别为0。此外,这两项当节点2在时间步1和2与节点1在时间步2共线或者节点1在时间步1和2与节点2在时间步1共线时为0.
%\section{结论}
%略
%\begin{thebibliography}{1}
%\item{略}
%\end{thebibliography}
%证明思路,找到一个数,对这个数求逆
%\includepdf{page5.pdf}
%\end{document}
%% \documentclass[10pt,conference]{IEEEtran}
% \usepackage{xeCJK}%preamble part
% \usepackage{graphicx}
% \usepackage{indentfirst}
% \usepackage{enumerate}
% \usepackage[a4paper, inner=1.5cm, outer=3cm, top=2cm, bottom=3cm, bindingoffset=1cm]{geometry}
% \usepackage{epstopdf}
% \usepackage{listings}
% \usepackage{multicol}
% \usepackage{array}
% \usepackage{fontspec}
% \usepackage{bm}
% \usepackage{gensymb}
% \usepackage{todonotes}
% \usepackage{amsmath, amsthm, amssymb}
% \usepackage[citecolor=blue]{hyperref}
% \newtheorem{definition}{Definition}
% \newtheorem{thm}{Theorem}[section]
% \newtheorem{thm}{Theorem}
% \newtheorem{cor}{Corollary}
% \newtheorem{proposition}{Proposition}
% \newtheorem{lem}{Lemma}
% \newtheorem{remark}{Remark}
% \DeclareMathOperator{\sgn}{sgn}
% \theoremstyle{remark}
% \newtheorem*{rem}{Remark}
% \setCJKmainfont[BoldFont={SimHei}]{SimSun}
% \setCJKmonofont{SimSun}
% \setmainfont{Times New Roman}
% \newCJKfontfamily[hei]\heiti{SimHei}
% \setlength{\extrarowheight}{4pt}
% \setlength{\parindent}{1cm}
 
% \begin{document}
% \onecolumn
% \title{宽带定位的理论极限--第二部分:协作网络} 
% \author{\fontsize{12pt}{\baselineskip}{沈渊,清华大学电子工程系副教授}}
% \maketitle
% \begin{multicols}{2}
% \begin{abstract}
\title{第三篇--宽带定位的理论极限---第二部分:协作网络(节选)$^{[3]}$}
%第三篇论文的翻译,原文的题目是
%Fundamental Limits Of Wideband Localization--Part II: Cooperative Networks
%由于原文较长,翻译部分只针对原文关于EFIM的一阶上下界的部分。
%\end{abstract}


对于每个移动节点,EFIM的准确表达式非常复杂。但是我们可以找到每个节点EFIM的上下界,从而获得对定位问题的洞见。

\begin{proposition}
设$\bm{J}_e^A(\bm{p}_k)=\bm{F}(\mu_k,\eta_k,\vartheta_k)$表示节点k从锚点获得的定位信息,$\bm{C}_{kj}=\bm{F}(\nu_{kj},0,\phi_{kj})$表示该节点和节点j协作的测距信息RI。节点k的EFIM $\bm{J}_e(\bm{p}_k)$满足如下不等式:
\[
\bm{J}_e^L(\bm{p}_k) \preceq \bm{J}_e(\bm{p}_k) \preceq \bm{J}_e^U(\bm{p}_k)
\]
其中
\begin{eqnarray}
\bm{J}_e^L(\bm{p}_k)=\bm{J}_e^A(\bm{p}_k)+\sum_{j\in \mathcal{N}_a \backslash {k}} \xi_{kj}^L \bm{C}_{kj}\\
\bm{J}_e^U(\bm{p}_k)=\bm{J}_e^A(\bm{p}_k)+\sum_{j\in \mathcal{N}_a \backslash {k}} \xi_{kj}^U \bm{C}_{kj}
\end{eqnarray}
\end{proposition}
%\end{multicols}
%\hrulefill
\begin{proof}
不失一般性,我们假设k=1.

下界:考虑EFIM$\bm{J}_e^L(\bm{P})$为:
\begin{equation}
\bm{J}_e^L(\bm{P})=\left[
\begin{array}{cccc}
\bm{J}_e^A(\bm{p}_1)+\sum_{j\in \mathcal{N}_a\backslash \{1\}}\bm{C}_{1,j}&-\bm{C}_{1,2}& \dots & -\bm{C}_{1,N_a}\\
-\bm{C}_{1,2} &\bm{J}_e^A(\bm{p}_2)+\bm{C}_{1,2} & \dots & 0\\
\vdots & \vdots & \ddots & \vdots\\
-\bm{C}_{1,N_a} & 0 & \dots & \bm{J}_e^A(\bm{p}_{N_a})+\bm{C}_{1,N_a}
\end{array}
\right]
\end{equation}
%\hrulefill
%\begin{multicols}{2}
这个矩阵是令$\bm{J}_e(\bm{P})$中所有的$\bm{C}_{kj}=0$,对于$1\le k,j \leq N_a$.这个EFIM对应着节点2到$N_a$的协作完全被忽略。通过线性代数的指数可以证明$\bm{J}_e^L(\bm{P})\preceq \bm{J}_e^L(\bm{P})$,这也可直观相符,因为没有利用节点2到$N_a$的协作信息。利用EFI的方法,我们可以得到节点1的EFIM是:
\begin{equation*}
\begin{split}
&\bm{J}_e^L(\bm{p}_1)=\bm{J}_e^A(\bm{p}_1)\\
&+\sum_{j\in \mathcal{N}_a \backslash \{1\}}[\bm{C}_{1,j}-\bm{C}_{1,j}(\bm{J}_e^A(\bm{p}_j)+\bm{C}_{1,j})^{-1}\bm{C}_{1,j}]
\end{split}
\end{equation*}
因为$\bm{C}_{1,j}=\nu_{1,j}\bm{q}_{\phi_{1,j}}\bm{q}_{\phi_{1,j}}^T$,其中$\bm{q}_{\phi_{1,j}}\triangleq=[\cos(\phi_{1,j}),\sin(\phi_{1,j})]^T$,我们可以将$\bm{J}^L_e(\bm{p}_1)$表示成:
\begin{equation}
\bm{J}_e^L(\bm{p}_1)=\bm{J}_e^A(\bm{p}_1)
+\sum_{j\in \mathcal{N}_a \backslash \{1\}}\xi_{1,j}^L\bm{C}_{1,j}
\end{equation}
其中$\xi_{1,j}^L\triangleq 1-\nu_{1,j}\bm{q}^T_{\phi_{1,j}}(\bm{J}^A_e(\bm{p}_j)+\bm{C}_{1,j})^{-1}\bm{q}_{\phi_{1,j}}$.系数$\xi_{1,j}^L$可以进一步化简为
\begin{equation}
\begin{split}
\xi_{1,j}^L=&1-\nu_{1,j}\bm{q}^T_{\vartheta_j-\phi_{1,j}}\\
&\cdot (\text{diag}\{\mu_j,\eta_j\}+\nu_{1,j}\bm{q}_{\vartheta_j-\phi_{1,j}}\bm{q}^T_{\vartheta_j-\phi_{1,j}})^{-1}\bm{q}_{\vartheta_j-\phi_{1,j}}\\
=&\frac{1}{1+\nu_{1,j}\delta_j(\phi_{1,j})}
\end{split}
\end{equation}
其中
$$\delta_j(\phi_{1,j})=\frac{1}{\mu_j}\cos^2(\vartheta-\phi_{1,j})+\frac{1}{\eta_j}\sin^2(\vartheta-\phi_{1,j})$$

上界: 考虑EFIM$\bm{J}_e^U(\bm{P})$为:
%\end{multicols}
%\hrulefill
\begin{equation}
\bm{J}_e^U(\bm{P})=\left[
\begin{array}{cccc}
\bm{J}_e^A(\bm{p}_1)+&-\bm{C}_{1,2}& \dots & -\bm{C}_{1,N_a}\\
\sum_{j\in \mathcal{N}_a\backslash \{1\}}\bm{C}_{1,j}&&&\\
-\bm{C}_{1,2} &\bm{J}_e^A(\bm{p}_2)& \dots & 0\\
&+\bm{C}_{1,2}+\sum_{j\in \mathcal{N}_a\backslash \{1,2\}}2\bm{C}_{1,j} &&\\
\vdots & \vdots & \ddots & \vdots\\
-\bm{C}_{1,N_a} & 0 & \dots & \bm{J}_e^A(\bm{p}_{N_a})\\
&&&+\sum_{j\in \mathcal{N}_a\backslash \{1\}}\bm{C}_{1,j}2\bm{C}_{1,N_a}\\
\end{array}
\right]
\end{equation}
%\hrulefill
%\begin{multicols}{2}
这个矩阵可以通过把$\bm{J}_e^(\bm{P})$种对角元$\bm{C}_{kj}$扩大一倍同时让非对角元$-\bm{C}_{kj}=0$,对$1\le k,j\leq N_a$.用线性代数的知识我们可以证明$\bm{J}_e^U(\bm{P}) \succeq \bm{J}_e(\bm{P})$,这也符合直观,因为节点2到$N_a$之间有了更多的协作。利用EFI的方法,我们可以得到节点1的EFIM是:
\begin{equation*}
\bm{J}_e^U(\bm{p}_1)=\bm{J}_e^A(\bm{p}_1)
+\sum_{j\in \mathcal{N}_a \backslash \{1\}}\xi_{1,j}^U\bm{C}_{1,j}
\end{equation*}
其中
\begin{equation}
\xi_{1,j}^U =\frac{1}{1+\nu_{1,j}\tilde{\delta}_j(\phi_{1,j})}
\end{equation}
而$$\tilde{\delta}_j(\phi_{1,j})=\frac{1}{\tilde{\mu}_j}\cos^2(\tilde{\vartheta}-\phi_{1,j})+\frac{1}{\tilde{\eta}_j}\sin^2(\tilde{\vartheta}-\phi_{1,j})$$
而$\tilde{\mu}_j,\tilde{\eta}_j,\tilde{\vartheta}_j$满足:
$$
\bm{F}(\tilde{\mu}_j,\tilde{\eta}_j,\tilde{\vartheta}_j)=\bm{J}^A_e(\bm{p}_j)+\sum_{k\in \mathcal{N}_a\backslash \{1,j\}} 2\bm{C}_{jk}
$$
%\end{multicols}
\end{proof}
%\end{document}
%\end{appendix}
%\end{document}
\newpage
% \documentclass[10pt,journal,compsoc]{IEEEtran}
% \usepackage{xeCJK}%preamble part
% \usepackage{graphicx}
% \usepackage{pdfpages}
% \usepackage{indentfirst}
% \usepackage{enumerate}
% \usepackage[a4paper, inner=1.5cm, outer=3cm, top=2cm, bottom=3cm, bindingoffset=1cm]{geometry}
% \usepackage{epstopdf}
% \usepackage{listings}
% \usepackage{array}
% \usepackage{fontspec}
% \usepackage{bm}
% \usepackage{gensymb}
% \usepackage{todonotes}
% \usepackage{amsmath, amsthm, amssymb}
% \usepackage[citecolor=blue]{hyperref}
% \newtheorem{definition}{Definition}
%\newtheorem{thm}{Theorem}[section]
% \newtheorem{thm}{Theorem}
% \newtheorem{corollary}{Corollary}
% \newtheorem{lem}{Lemma}
% \newtheorem{remark}{Remark}
% \newtheorem{proposition}{Proposition}
% \DeclareMathOperator{\sgn}{sgn}
% \theoremstyle{remark}
% \newtheorem*{rem}{Remark}
% \setCJKmainfont[BoldFont={SimHei}]{SimSun}
% \setCJKmonofont{SimSun}
% \setmainfont{Times New Roman}
% \newCJKfontfamily[hei]\heiti{SimHei}
% \setlength{\extrarowheight}{4pt}
% \setlength{\parindent}{1cm}
 
%\begin{document}
\title{第二篇--协作定位网络中的时空信息耦合$^{[2]}$} 
%\author{\fontsize{12pt}{\baselineskip}{沈渊,清华大学电子工程系副教授}}
%\markboth{Journal of \LaTeX\ Class Files,~Vol.~14, No.~8, August~2015}%
%{Shell \MakeLowercase{\textit{et al.}}: Bare Advanced Demo of IEEEtran.cls for IEEE Computer Society Journals}
%\IEEEtitleabstractindextext{
%\begin{abstract}
\textbf{摘要}:%第二篇论文翻译,原文的题目是
%Spatio-Temporal Information Coupling in Cooperative Network Navigation
%原文的摘要翻译如下:
可靠的定位信息对很多基于位置的应用起着至关重要的作用。通过时空联合协作的网络导航可以给移动节点提供高精度和鲁棒的位置信息。同时,由于待测节点的位置相关,这种联合协作导致了错综复杂的信息获取方式,也就是信息耦合的问题。在这篇文章中,通过对费舍尔信息量的分析,我们在四种有代表性的情形中量化了信息耦合。我们说明了每个节点所获得的信息来自与它进行时空协作的节点和由于和邻居节点的协作产生的信息耦合。我们的结果为网络中复杂信息获取提供了洞见,并且能够为高效的网络导航算法提供指导。
%\end{abstract}
%}\maketitle
%reset section counter here
\setcounter{section}{0}
\section{简介}
关于研究背景的翻译略去,下面是原文一些记号上的约定。
$[\cdot]^T$表示变量的转置;$\mathbb{S}^D,\mathbb{S}^D_+,\mathbb{S}^D_{++}$分别表示$D\times D$的实矩阵、半正定矩阵和正定矩阵。$\bm{J}_r(\bm{v}):=\bm{v}\bm{v}^T$表示由向量$\bm{v}$做外积得到的秩1阵。
另外$\angle \{ \bm{u},\bm{v}\}
$表示向量$\bm{u}$和向量$\bm{v}$之间的夹角。
\section{网络导航中的费舍尔信息量}
在本节中,我们首先介绍网络模型和网络导航中的FIM作为预备知识,然后描述这篇文章重点讨论的4种场景。
\subsection{预备知识}
考虑一个由若干节点构成的协作网络,用$\bm{x}_k^{(n)}\in \mathbb{R}^D$表示节点k在时间$t_n$的位置状态,$k=1,2,....,N$且$n=1,2,...T$.网络导航的目标是从测量和先验信息中推断节点的位置信息。
令$\mathcal{S}=\{1,2,...,S\}\text{ with }S=N\cdot T$是位置状态的下标集,文献[8]已经证明了对于S个位置状态FIM可以分解成:
\begin{equation}
\bm{J}=\sum_{(i,j)\in \mathcal{S}^2,i\geq j} \bm{G}^S_{i,j} \otimes \bm{K}_{i,j}
\end{equation}
其中$\otimes$表示Kronecker矩阵积,$S\times S$的矩阵$\bm{G}^S_{i,j}$的元素$(k,r)$由下式给出:
\[
[\bm{G}^S_{i,j}]_{k,r}=\left\{
\begin{array}{c l}	
     1 & k=r=i\\
     1 & k=r=j\\
     -1 & k=i,r=j,k \neq r\\
     -1 & k=j,r=i,k \neq r\\
     0, & \text{otherwise}
\end{array}\right.
\]
$\bm{K}_{i,j}\in \mathbb{S}_+^D$ 描述了包含在测量或先验知识中和位置状态i,j有关的位置信息。$\bm{K}_{i,j}=\bm{K}_{j,i}$并且在缺少关于位置状态i,j的测量或先验知识的情况下$\bm{K}_{i,j}=0$。特别的,如果$i=j$,$\bm{K}_{i,i}$描述了仅仅和位置状态i有关的位置信息。
\subsection{有代表性的场景}
我们考虑图1中的四种场景以获得对时空协作的洞见。
\begin{figure}
\centering
\includegraphics[width=\textwidth/2]{figure_1.png}
\caption{四种场景:(A)两个节点的空间协作(红色的连接);(B)三个节点的空间协作;(C)一个节点的时间协作(绿色的连接);(D)两个节点的时空协作。}
\end{figure}

在这些场景中,位置状态就是节点的位置,节点通过和位置已知的锚点和位置未知的邻居节点的测距中获得位置信息。
在接下来的讨论中,在时间步为n时,$\bm{K}^{(n)}_{i,i}$表示节点i从锚点获取的信息;$\bm{u}_{i,j}^{(n)}$表示节点i和j连线的单位方向向量;$\lambda^{(n)}_{i,j}=\lambda^{(n)}_{j,i}$表示节点i和j之间的测距信息强度(RII);$\bm{u}_{i,i}^{(n)}$表示链接节点i在时间步为n和n+1时的位置的单位方向向量;$\lambda^{(n)}_{i,i}$表示节点i的速度信息强度(SII)。

\textbf{场景A:}两个节点通过彼此测距协作来确定它们的位置,即位置状态是$\bm{p}_1$和$\bm{p}_2$,测距信息是$\bm{J}_r(\bm{v}_{1,2})$,其中$\bm{v}_{1,2}=\sqrt{\lambda_{1,2}}\bm{u}_{1,2}$,因此,基于(1)式我们得到这种场景的FIM是:
\begin{equation}
\bm{J}_A=\left[
\begin{array}{cc}
\bm{K}_{1,1}+\bm{J}_r(\bm{v}_{1,2})&-\bm{J}_r(\bm{v}_{1,2})\\
-\bm{J}_r(\bm{v}_{1,2})&\bm{K}_{2,2}+\bm{J}_r(\bm{v}_{1,2})\\
\end{array}
\right]
\end{equation}
%\begin{figure*}[!t]
\begin{equation*}
\bm{J}_B=\left[
\begin{array}{ccc}
\bm{K}_{1,1}+\bm{J}_r(\bm{v}_{1,2})+\bm{J}_r(\bm{v}_{1,2}) & -\bm{J}_r(\bm{v}_{1,2}) & -\bm{J}_r(\bm{v}_{1,3})\\
-\bm{J}_r(\bm{v}_{1,2}) &\bm{K}_{2,2}+\bm{J}_r(\bm{v}_{1,2})+\bm{J}_r(\bm{v}_{2,3}) & -\bm{J}_r(\bm{v}_{2,3})\\
 -\bm{J}_r(\bm{v}_{1,3}) & -\bm{J}_r(\bm{v}_{2,3})&
\bm{K}_{3,3}+\bm{J}_r(\bm{v}_{1,3})+\bm{J}_r(\bm{v}_{2,3})\\
\end{array}
\right]
\end{equation*}
%\hrulefill
%\vspace*{4pt}
%\end{figure*}

\textbf{场景B:} 三个节点通过彼此测距协作确定它们的位置,即位置状态是$\bm{p}_1,\bm{p}_2$和$\bm{p}_3$,测距信息是$\bm{J}_r(\bm{v}_{i,j})$对于$i,j\in \{1,2,3\},i\neq j$,其中$\bm{v}_{i,j}=\sqrt{\lambda_{i,j}}\bm{u}_{i,j}$,因此,基于式(1)我们可以得到这种情形的FIM由(3)给出,在下一页的最开始的地方。

\textbf{场景C:}单节点在两个不同的时刻通过速度测量协作来确定自身的位置,即位置状态是$\bm{p}^{(1)}_1$和$\bm{p}^{(2)}_1$,速度测量的信息是$\bm{J}_r(\bm{t})$,其中$\bm{t}=\sqrt{\lambda^{(1)}_{1,1}}\bm{u}_{1,1}^{(1)}$.因此基于(1),我们得到这种场景的FIM是:
\[\bm{J}_C=\left[
\begin{array}{cc}
\bm{K}_{1,1}^{(1)}+\bm{J}_r(\bm{t})&-\bm{J}_r(\bm{t})\\
-\bm{J}_r(\bm{t})&\bm{K}_{2,2}^{(2)}+\bm{J}_r(\bm{t})\\
\end{array}\right]
\]

\textbf{场景D:}两个节点在两个不同的时刻协作来确定它们的位置,即,位置状态是$\bm{p}_1^{(1)},\bm{p}_2^{(1)},\bm{p}_1^{(2)}$和$\bm{p}_2^{(2)}$,每个节点测量自身的速度和相对另一节点的距离,并且
\begin{itemize}  
\item 测距信息是$\bm{J}_r(\bm{v}_{1,2})$和$\bm{J}_r(\bm{w}_{i,j})$,其中$\bm{v}_{1,2}=\sqrt{\lambda_{1,2}^{(1)}}\bm{u}^{(2)}_{1,2}$,而$\bm{w}_{1,2}=\sqrt{\lambda_{1,2}^{(2)}}\bm{u}^{(1)}_{1,2}$
\item 测速信息是$\bm{J}_r(\bm{t}_1)$和$\bm{J}_r(\bm{t}_2)$,其中$\bm{t}_i=\sqrt{\lambda_{i,i}^{(1)}}\bm{u}^{(1)}_{i,i}$
\end{itemize}
因此,基于(1)式我们得到这种类形的FIM是
\[
\bm{J}_D=\left[
\begin{array}{cc}
\bm{J}^{(1)}_A+\bm{T}&-\bm{T}\\
-\bm{T}&\bm{J}^{(2)}_A+\bm{T}\\
\end{array}
\right]
\]
其中$\bm{J}^{(1)}_A$和$\bm{J}^{(2)}_A$由(2)式给出,上标代表了时间步长,$\bm{T}=\text{diag}\{\bm{J}_r(\bm{t}_1),\bm{J}_r(\bm{t}_2)\}$.
\section{空间信息耦合}
在这一节中,我们推导了场景A和场景B的EFIM,并且描述了场景B中由于空间协作导致的信息耦合。由于篇幅所限,这篇文章的大部分证明略去。
\begin{proposition}
场景A中节点1的EFI是
\[
\bm{J}_e=\bm{K}_{1,1}+(1-\mu_2^{1,1})\bm{J}_r(\bm{v}_{1,2})
\]
其中
\begin{eqnarray*}
\mu_2^{1,1}=&\bm{v}^T_{1,2}(\bm{K}_{2,2}+\bm{J}_r(\bm{v}_{1,2}))^{-1}\bm{v}_{1,2}\\
=&1-(1+\lambda_{1,2}\bm{u}^T_{1,2}\bm{K}^{-1}_{2,2}\bm{u}_{1,2})^{-1}.
\end{eqnarray*}
此外,$\bm{J}_e \preceq \bm{K}_{1,1}+\bm{J}_r(\bm{v}_{1,2}).$
\end{proposition}
\begin{remark}
场景A中节点1的EFIM是来自锚点和来自与节点2的协作的信息之和。协作在与节点2的连线方向上增加了信息,而节点1可以有效利用的RII随RII $\lambda_{1,2}$增大而增大而随节点2位置在$\bm{u}_{1,2}$方向的不确定性(即$\bm{u}^T\bm{K}^{-1}_{2,2}\bm{u}_{1,2}$)的增大而减小。此外,有效RII总是小于或等于RII。当锚点在节点1,2连线方向上给节点2提供无穷大的信息量时取等号,而当锚点
在这一方向上不提供任何信息时取0。
\end{remark}
\begin{proposition}
在场景B中节点1的EFIM是
\begin{equation}
\begin{split}
\bm{J}_e=\bm{K}_{1,1}+(1-\mu_2^{1,1})\bm{J}_r(\bm{v}_{1,2})+(1-\mu_3^{1,1})\bm{J}_r(\bm{v}_{1,3})\\
+\delta\bm{J}_r(\mu_2^{1,3}\bm{v}_{1,2}+\mu_3^{1,2}\bm{v}_{1,3})
\end{split}
\end{equation}
其中$\mu_i^{j,k}=\bm{v}^T_{j,i}(\bm{K}_{i,i}+\bm{J}_r(\bm{v}_{1,i}))^{-1}\bm{v}_{k,i}$,并且$\delta=(1+\mu_2^{3,3}+\mu_3^{2,2})^{-1}$.除此之外,$\bm{J}_e \preceq \bm{K}_{1,1}+\bm{J}_r(\bm{v}_{1,2})+\bm{J}_r(\bm{v}_{1,3})$.
\end{proposition}
\begin{remark}
命题2说明了场景B下节点1的EFIM是4项的和。第一项对应着从锚点获取的信息,其他三项对应着由于空间协作获得的信息。第二项和第三项对应着沿着与节点2和节点3的连线方向获取的有效信息,而最后一项代表着由于节点2和3协作造成的信息耦合。最后一项是一个秩一阵,有一个非负的特征值$\delta ||\mu_2^{1,3}\bm{v}_{1,2}+\mu_3^{1,2}\bm{v}_{1,3}||^2$并且这个特征值对应的特征向量是$\mu_2^{1,3}\bm{v}_{1,2}+\mu_3^{1,2}\bm{v}_{1,3}$,这个方向既依赖于RII,又依赖于节点2和3位置的不确定性,还和节点的空间拓扑有关。
\end{remark}
接下来我们将说明节点2和3的协作强度如何影响节点1的EFIM.
\begin{corollary}
设$\lambda_{2,3}$和$\tilde{\lambda}_{2,3}$分别表示节点2和节点3之间两个RII,并且$(\mu_2^{1,3},\mu_3^{1,2},\delta),(\tilde{\mu}_2^{1,3},\tilde{\mu}_3^{1,2},\tilde{\delta})$分别表示命题2中对应的两组参数,而其他参数均相同。那么:
\begin{equation}
\delta \bm{J}_r(\mu_2^{1,3}\bm{v}_{1,2}+\mu_3^{1,2}\bm{v}_{1,3}) \preceq \tilde{\delta} \bm{J}_r(\tilde{\mu}_2^{1,3}\bm{v}_{1,2}+\tilde{\mu}_3^{1,2}\bm{v}_{1,3})
\end{equation}
当且仅当$\lambda_{2,3}\leq \tilde{\lambda}_{2,3}$。此外
\[
\lim_{\lambda_{2,3} \rightarrow \infty}\delta \bm{J}_r(\mu_2^{1,3}\bm{v}_{1,2}+\mu_3^{1,2}\bm{v}_{1,3})=\bar{\delta} \bm{J}_r(\bar{\mu}_2^{1,3}\bm{v}_{1,2}+\bar{\mu}_3^{1,2}\bm{v}_{1,3})
\]
其中
\[
\bar{\mu}_2^{1,3}=\bm{v}_{1,2}^T(\bm{K}_{2,2}+\bm{J}_r(\bm{v}_{1,2}))^{-1}\bm{u}_{2,3}
\]
\[
\bar{\mu}_3^{1,2}=\bm{v}_{1,3}^T(\bm{K}_{3,3}+\bm{J}_r(\bm{v}_{1,3}))^{-1}\bm{u}_{2,3}
\]
\[
\bar{\delta}=(\bm{u}_{2,3}^T((\bm{K}_{2,2}+\bm{J}_r(\bm{v}_{1,2}))^{-1}+(\bm{K}_{3,3}+\bm{J}_r(\bm{v}_{1,3}))^{-1})\bm{u}_{2,3})^{-1}.
\]
\end{corollary}
\begin{remark}
从(5)中可以知道,当节点2,3方向保持不变时,节点1从和节点2和3协作中获得的信息随着节点2和3之间的RII的增大而增大。另外,这个信息量的上界是在节点2和3之间测距是理想情况时取得。
\end{remark}
在下面这个推论中,我们将说明如果从协作中获得的信息是分别和节点2,节点3的测距信息的加权和这种情形。
\begin{corollary}
场景B中节点1的EFIM可以写成
\[
\bm{J}_e=\bm{K}_{1,1}+\eta_1 \bm{J}_r(\bm{v}_{1,2})+\eta_2 \bm{J}_r(\bm{v}_{1,3}),\eta_1,\eta_2 \geq 0
\]
当且仅当至少下面至少有一个条件满足:
\begin{itemize}
\item{三个节点共线}
\item $\lambda_{1,2}\cdot \lambda_{1,3}\cdot \lambda_{2,3}=0$
\item $\bm{u}_{1,2}^T(\bm{K}_{2,2}+\bm{J}_r(\bm(v)_{1,2}))^{-1}\bm{u}_{2,3}=0$
\item $\bm{u}_{1,3}^T(\bm{K}_{3,3}+\bm{J}_r(\bm(v)_{1,3}))^{-1}\bm{u}_{2,3}=0$
\end{itemize}
\end{corollary}
接下来我们将说明如果锚点提供各向同性的信息给节点2,3,那么权重参数$\mu_i^{j,k}$可以写成$\theta_1=\angle \{\bm{u}_{1,2},\bm{u}_{2,3}\}$和$\theta_2=\angle \{\bm{u}_{1,3},\bm{u}_{2,3}\}$的函数。
\begin{corollary}
如果$\bm{K}_{i,i}=\xi_i\bm{I}$对于$i=2,3$成立,则场景B中节点1的EFIM为
\begin{equation*}
\begin{split}
\bm{J}_e=\bm{K}_{1,1}+\frac{\xi_2}{\xi_2+\lambda_{1,2}}\bm{J}_r(\bm{v}_{1,2})+\frac{\xi_3}{\xi_3+\lambda_{1,3}}\bm{J}_r(\bm{v}_{1,3})\\
+\delta\bm{J}_r(\mu_2^{1,3}\bm{v}_{1,2}+\mu_3^{1,2}\bm{v}_{1,3})
\end{split}
\end{equation*}
其中
\[
\mu_2^{1,3}=\frac{\sqrt{\lambda_{1,2}\lambda_{2,3}}\cos \theta_1}{\xi_2+\lambda_{1,2}}
\]
\[
\mu_3^{1,2}=\frac{\sqrt{\lambda_{1,3}\lambda_{2,3}}\cos \theta_2}{\xi_3+\lambda_{1,3}}
\]
\[
\mu_2^{3,3}=\frac{\lambda_{2,3}}{\xi_2}(1-\frac{\lambda_{1,2}\cos^2 (\theta_2)}{\xi_2+\lambda_{1,2}})
\]
\[
\mu_3^{2,2}=\frac{\lambda_{2,3}}{\xi_3}(1-\frac{\lambda_{1,3}\cos^2 (\theta_2)}{\xi_3+\lambda_{1,3}})
\]
\end{corollary}
\begin{remark}
这个结果说明了如果$(i)$节点2和3来自锚点或先验的信息是各向同性的$(ii)\bm{u}_{1,2}\perp\bm{u}_{2,3}$或者$\bm{u}_{1,3}\perp\bm{u}_{2,3}$,那么节点1通过写作获得的信息是分别来自节点2的测距信息的加权和。
\end{remark}
\section{空时信息耦合}
在本节中,我们推导出了场景C和D下节点1在时间步2时的EFIM,并且描述了场景D中由于空时协作产生的信息耦合。
\begin{proposition}
场景C中节点1的EFIM是
\[
\bm{J}_e=\bm{K}_{1,1}^{(2)}+\frac{\lambda_{1,1}^{(1)}}{1+\lambda_{1,1}^{(1)}\bm{u}_{1,1}^T(\bm{K}_{1,1}^{1})^{-1}\bm{u}_{1,1}}\bm{J}_r(\bm{u}_{1,1})
\]
此外$\bm{J}_e \preceq \bm{K}_{1,1}^{(2)}+\bm{J}_r(\bm{t})$.
\end{proposition}
\begin{remark}
与场景A类似,场景C中时间步为2时节点1的EFIM是从锚点获取的信息和与节点1在时间步为1时的协作信息之和。协作在节点1在两个时刻位置连线的方向上增加了信息,而节点1可以有效利用的SII随SII $\lambda_{1,1}^{(1)}$的增加而增大而随着节点1在时间步1的位置不确定性(即方向$\bm{u}_{1,1}^T(\bm{K}_{1,1}^{(1)})^{-1}\bm{u}_{1,1}$)的增大而减小。此外,有效的SII总是小于等于SII,当锚点给时间步1时的节点提供无穷大的信息时取等号,当不提供任何信息时取0.
\end{remark}
\begin{proposition}
场景D中节点1在时间步2时的EFIM是
\begin{equation}
\begin{split}
\bm{J}_e=\bm{K}_{1,1}^{(2)}+&[1-\bm{v}_{1,2}^T(\bm{K}_{2,2}^{(2)}+\bm{H}_{1,2})^{-1}\bm{v}_{1,2}]\bm{J}_r(\bm{v}_{1,2})\\
+&[1-\bm{t}_1^T(\bm{K}_{1,1}^{(1)}+\bm{H}_{1,1})^{-1}\bm{t}_1]\bm{J}_r(\bm{t}_1)\\
+&\delta \bm{J}_r(\nu_2 \bm{v}_{1,2}+\nu_1 \bm{t}_1)
\end{split}
\end{equation}
此外$\bm{J}_e \preceq \bm{K}_{1,1}^{(2)}+\bm{J}_r(\bm{v}_{1,2})+\bm{J}_r(\bm{t}_1)$
\end{proposition}
\begin{remark}
和场景B类似,命题4说明了在场景D中,时间步2时节点1的EFIM是4项的和。第一项对应着从锚点获得的信息,其余的项对应着通过协作获得的信息。第二和第三项分别是节点2在时间步2和节点1进行空间协作以及和自身在时间步1的时间协作获得的信息。而最后一项对应着耦合项。最后一项是一个秩一阵,有一个非负的特征值为$\delta||\nu_2\bm{v}_{1,2}+\nu_1\bm{t}_1||^2$,对应的特征向量为$\nu_2\bm{v}_{1,2}+\nu_1\bm{t}_1$。这个方向依赖于RII,SII,协作节点位置的不确定度以及节点在空间上的拓扑。特别的,当$\bm{t}_2^T(\bm{K}_{2,2}^{(1)})^{-1}\bm{w}_{1,2}=0$时,比如$\bm{K}_{2,2}^{(1)}=\xi_{2,2}^{(1)}\bm{I}$且$\bm{u}_{1,2}^{(1)} \perp\bm{u}_{2,2}^{(1)}$,最后一项为0.
\end{remark}
\section{数值结果}
在本节中,我们给出若干个含有信息耦合的EFIM的数值的例子。特别的,我们将考察节点的网络拓扑如何影响信息耦合项。

信息耦合项是协作节点方向加权组合作外积得到的,对于场景B,权系数是$\sqrt{\delta}\mu_2^{1,3}$和
$\sqrt{\delta}\mu_3^{1,2}$,对于场景D权系数是$\sqrt{\delta}\nu_2$和
$\sqrt{\delta}\nu_1$,图2和图3分别说明了这些系数对于角度$\theta_1=\angle \{\bm{u}_{1,2},\bm{u}_{2,3}\},\theta_2=\angle \{\bm{u}_{1,3},\bm{u}_{2,3}\}$或者$\vartheta_1=\angle \{\bm{u}_{1,1},\bm{u}_{1,2}^{(1)}\},\vartheta_2=\angle \{\bm{u}_{2,2},\bm{u}_{1,2}^{(2)}\}$的依赖关系。在数值结果里,我们设
$\bm{K}_{2,2}=\bm{K}_{3,3}=\bm{K}_{2,2}^{(2)}=\bm{K}_{2,2}^{(1)}=\bm{K}_{1,1}^{(1)}=\bm{I}$,所以的RII和SII都取1,并且$\bm{u}_{1,1}^T\bm{u}_{1,2}^{(2)}=0$.

在图2中,我们可以看到$\sqrt{\delta}\mu_2^{1,3}$和$\sqrt{\delta}\mu_3^{1,2}$,即$\bm{v}_{1,2}$或$\bm{v}_{1,3}$的权重,当$\bm{u}_{1,2}$和$\bm{u}_{1,3}$正交于$\bm{u}_{2,3}$时分别为0.此外,这两项在节点1,2,3共线时达到最大值。

类似的,在图3中,我们可以看到$\sqrt{\delta}\nu_2$和$\sqrt{\delta}\nu_1$,即耦合项中$\bm{v}_{1,2}$或$\bm{t}_1$的系数,当$\bm{u}_{2,2}$正交于$\bm{u}_{1,2}^{(2)}$且$\bm{u}_{1,1}$正交于$\bm{u}_{1,2}^{(1)}$时分别为0。此外,这两项当节点2在时间步1和2与节点1在时间步2共线或者节点1在时间步1和2与节点2在时间步1共线时为0.
%\section{结论}
%略
%\begin{thebibliography}{1}
%\item{略}
%\end{thebibliography}
%证明思路,找到一个数,对这个数求逆
%\includepdf{page5.pdf}
%\end{document}
% \documentclass[10pt,conference]{IEEEtran}
% \usepackage{xeCJK}%preamble part
% \usepackage{graphicx}
% \usepackage{indentfirst}
% \usepackage{enumerate}
% \usepackage[a4paper, inner=1.5cm, outer=3cm, top=2cm, bottom=3cm, bindingoffset=1cm]{geometry}
% \usepackage{epstopdf}
% \usepackage{listings}
% \usepackage{multicol}
% \usepackage{array}
% \usepackage{fontspec}
% \usepackage{bm}
% \usepackage{gensymb}
% \usepackage{todonotes}
% \usepackage{amsmath, amsthm, amssymb}
% \usepackage[citecolor=blue]{hyperref}
% \newtheorem{definition}{Definition}
% \newtheorem{thm}{Theorem}[section]
% \newtheorem{thm}{Theorem}
% \newtheorem{cor}{Corollary}
% \newtheorem{proposition}{Proposition}
% \newtheorem{lem}{Lemma}
% \newtheorem{remark}{Remark}
% \DeclareMathOperator{\sgn}{sgn}
% \theoremstyle{remark}
% \newtheorem*{rem}{Remark}
% \setCJKmainfont[BoldFont={SimHei}]{SimSun}
% \setCJKmonofont{SimSun}
% \setmainfont{Times New Roman}
% \newCJKfontfamily[hei]\heiti{SimHei}
% \setlength{\extrarowheight}{4pt}
% \setlength{\parindent}{1cm}
 
% \begin{document}
% \onecolumn
% \title{宽带定位的理论极限--第二部分:协作网络} 
% \author{\fontsize{12pt}{\baselineskip}{沈渊,清华大学电子工程系副教授}}
% \maketitle
% \begin{multicols}{2}
% \begin{abstract}
\title{第三篇--宽带定位的理论极限---第二部分:协作网络(节选)$^{[3]}$}
%第三篇论文的翻译,原文的题目是
%Fundamental Limits Of Wideband Localization--Part II: Cooperative Networks
%由于原文较长,翻译部分只针对原文关于EFIM的一阶上下界的部分。
%\end{abstract}


对于每个移动节点,EFIM的准确表达式非常复杂。但是我们可以找到每个节点EFIM的上下界,从而获得对定位问题的洞见。

\begin{proposition}
设$\bm{J}_e^A(\bm{p}_k)=\bm{F}(\mu_k,\eta_k,\vartheta_k)$表示节点k从锚点获得的定位信息,$\bm{C}_{kj}=\bm{F}(\nu_{kj},0,\phi_{kj})$表示该节点和节点j协作的测距信息RI。节点k的EFIM $\bm{J}_e(\bm{p}_k)$满足如下不等式:
\[
\bm{J}_e^L(\bm{p}_k) \preceq \bm{J}_e(\bm{p}_k) \preceq \bm{J}_e^U(\bm{p}_k)
\]
其中
\begin{eqnarray}
\bm{J}_e^L(\bm{p}_k)=\bm{J}_e^A(\bm{p}_k)+\sum_{j\in \mathcal{N}_a \backslash {k}} \xi_{kj}^L \bm{C}_{kj}\\
\bm{J}_e^U(\bm{p}_k)=\bm{J}_e^A(\bm{p}_k)+\sum_{j\in \mathcal{N}_a \backslash {k}} \xi_{kj}^U \bm{C}_{kj}
\end{eqnarray}
\end{proposition}
%\end{multicols}
%\hrulefill
\begin{proof}
不失一般性,我们假设k=1.

下界:考虑EFIM$\bm{J}_e^L(\bm{P})$为:
\begin{equation}
\bm{J}_e^L(\bm{P})=\left[
\begin{array}{cccc}
\bm{J}_e^A(\bm{p}_1)+\sum_{j\in \mathcal{N}_a\backslash \{1\}}\bm{C}_{1,j}&-\bm{C}_{1,2}& \dots & -\bm{C}_{1,N_a}\\
-\bm{C}_{1,2} &\bm{J}_e^A(\bm{p}_2)+\bm{C}_{1,2} & \dots & 0\\
\vdots & \vdots & \ddots & \vdots\\
-\bm{C}_{1,N_a} & 0 & \dots & \bm{J}_e^A(\bm{p}_{N_a})+\bm{C}_{1,N_a}
\end{array}
\right]
\end{equation}
%\hrulefill
%\begin{multicols}{2}
这个矩阵是令$\bm{J}_e(\bm{P})$中所有的$\bm{C}_{kj}=0$,对于$1\le k,j \leq N_a$.这个EFIM对应着节点2到$N_a$的协作完全被忽略。通过线性代数的指数可以证明$\bm{J}_e^L(\bm{P})\preceq \bm{J}_e^L(\bm{P})$,这也可直观相符,因为没有利用节点2到$N_a$的协作信息。利用EFI的方法,我们可以得到节点1的EFIM是:
\begin{equation*}
\begin{split}
&\bm{J}_e^L(\bm{p}_1)=\bm{J}_e^A(\bm{p}_1)\\
&+\sum_{j\in \mathcal{N}_a \backslash \{1\}}[\bm{C}_{1,j}-\bm{C}_{1,j}(\bm{J}_e^A(\bm{p}_j)+\bm{C}_{1,j})^{-1}\bm{C}_{1,j}]
\end{split}
\end{equation*}
因为$\bm{C}_{1,j}=\nu_{1,j}\bm{q}_{\phi_{1,j}}\bm{q}_{\phi_{1,j}}^T$,其中$\bm{q}_{\phi_{1,j}}\triangleq=[\cos(\phi_{1,j}),\sin(\phi_{1,j})]^T$,我们可以将$\bm{J}^L_e(\bm{p}_1)$表示成:
\begin{equation}
\bm{J}_e^L(\bm{p}_1)=\bm{J}_e^A(\bm{p}_1)
+\sum_{j\in \mathcal{N}_a \backslash \{1\}}\xi_{1,j}^L\bm{C}_{1,j}
\end{equation}
其中$\xi_{1,j}^L\triangleq 1-\nu_{1,j}\bm{q}^T_{\phi_{1,j}}(\bm{J}^A_e(\bm{p}_j)+\bm{C}_{1,j})^{-1}\bm{q}_{\phi_{1,j}}$.系数$\xi_{1,j}^L$可以进一步化简为
\begin{equation}
\begin{split}
\xi_{1,j}^L=&1-\nu_{1,j}\bm{q}^T_{\vartheta_j-\phi_{1,j}}\\
&\cdot (\text{diag}\{\mu_j,\eta_j\}+\nu_{1,j}\bm{q}_{\vartheta_j-\phi_{1,j}}\bm{q}^T_{\vartheta_j-\phi_{1,j}})^{-1}\bm{q}_{\vartheta_j-\phi_{1,j}}\\
=&\frac{1}{1+\nu_{1,j}\delta_j(\phi_{1,j})}
\end{split}
\end{equation}
其中
$$\delta_j(\phi_{1,j})=\frac{1}{\mu_j}\cos^2(\vartheta-\phi_{1,j})+\frac{1}{\eta_j}\sin^2(\vartheta-\phi_{1,j})$$

上界: 考虑EFIM$\bm{J}_e^U(\bm{P})$为:
%\end{multicols}
%\hrulefill
\begin{equation}
\bm{J}_e^U(\bm{P})=\left[
\begin{array}{cccc}
\bm{J}_e^A(\bm{p}_1)+&-\bm{C}_{1,2}& \dots & -\bm{C}_{1,N_a}\\
\sum_{j\in \mathcal{N}_a\backslash \{1\}}\bm{C}_{1,j}&&&\\
-\bm{C}_{1,2} &\bm{J}_e^A(\bm{p}_2)& \dots & 0\\
&+\bm{C}_{1,2}+\sum_{j\in \mathcal{N}_a\backslash \{1,2\}}2\bm{C}_{1,j} &&\\
\vdots & \vdots & \ddots & \vdots\\
-\bm{C}_{1,N_a} & 0 & \dots & \bm{J}_e^A(\bm{p}_{N_a})\\
&&&+\sum_{j\in \mathcal{N}_a\backslash \{1\}}\bm{C}_{1,j}2\bm{C}_{1,N_a}\\
\end{array}
\right]
\end{equation}
%\hrulefill
%\begin{multicols}{2}
这个矩阵可以通过把$\bm{J}_e^(\bm{P})$种对角元$\bm{C}_{kj}$扩大一倍同时让非对角元$-\bm{C}_{kj}=0$,对$1\le k,j\leq N_a$.用线性代数的知识我们可以证明$\bm{J}_e^U(\bm{P}) \succeq \bm{J}_e(\bm{P})$,这也符合直观,因为节点2到$N_a$之间有了更多的协作。利用EFI的方法,我们可以得到节点1的EFIM是:
\begin{equation*}
\bm{J}_e^U(\bm{p}_1)=\bm{J}_e^A(\bm{p}_1)
+\sum_{j\in \mathcal{N}_a \backslash \{1\}}\xi_{1,j}^U\bm{C}_{1,j}
\end{equation*}
其中
\begin{equation}
\xi_{1,j}^U =\frac{1}{1+\nu_{1,j}\tilde{\delta}_j(\phi_{1,j})}
\end{equation}
而$$\tilde{\delta}_j(\phi_{1,j})=\frac{1}{\tilde{\mu}_j}\cos^2(\tilde{\vartheta}-\phi_{1,j})+\frac{1}{\tilde{\eta}_j}\sin^2(\tilde{\vartheta}-\phi_{1,j})$$
而$\tilde{\mu}_j,\tilde{\eta}_j,\tilde{\vartheta}_j$满足:
$$
\bm{F}(\tilde{\mu}_j,\tilde{\eta}_j,\tilde{\vartheta}_j)=\bm{J}^A_e(\bm{p}_j)+\sum_{k\in \mathcal{N}_a\backslash \{1,j\}} 2\bm{C}_{jk}
$$
%\end{multicols}
\end{proof}
%\end{document}
\title{外文资料原文的索引}
\begin{translationbib}
\item S. Mazuelas, Y. Shen and M. Z. Win, "Information Coupling in Cooperative Localization," in IEEE Communications Letters, vol. 15, no. 7, pp. 737-739, July 2011.
\item S. Mazuelas, Y. Shen and M. Z. Win, "Spatio-temporal information coupling in cooperative network navigation," 2012 IEEE Global Communications Conference (GLOBECOM), Anaheim, CA, 2012, pp. 2403-2407.
\item Y. Shen, H. Wymeersch and M. Z. Win, "Fundamental Limits of Wideband Localization— Part II: Cooperative Networks" in IEEE Transactions on Information Theory, vol. 56, no. 10, pp. 4981-5000, Oct. 2010.
\end{translationbib}
\chapter{公式的推导}
\section{建模过程的一些推导过程}
\subsection{定位问题中费舍尔信息矩阵一般结构推导}\label{A_F_1}
在非协作单节点定位中,测量量的联合概率分布由式(\ref{eq:single})给出,费舍尔信息矩阵是费舍尔信息量的自然推广,在满足一定正则性的条件下,费舍尔信息矩阵可以写成:
\begin{equation}
I(\bm{p})=-\mathbb{E}_{\bm{x}}(\bigtriangledown_{\bm{p}} \log f(\vec{x}|\bm{p}))^{\textrm{T}}(\bigtriangledown_{\bm{p}} \log f(\vec{x}|\bm{p}))
\end{equation}
其中f是随机向量$\vec{x}$的密度函数,利用上面的公式,首先对式(\ref{eq:single})取对数并求梯度得:
\begin{equation}
\bigtriangledown_{\bm{p}}\ln f=-\sum_{i=1}^{N_b}\frac{||\bm{p}_i^b-\bm{p}||-x_i}{\sigma_i^2}\frac{\bm{p}^b_i-\bm{p}}{||\bm{p}^b_i-\bm{p}||}.
\end{equation}
注意到$\frac{||\bm{p}_i^b-\bm{p}||-x_i}{\sigma_i}\sim N(0,1)$,所以按照费舍尔信息矩阵的定义可得到式(\ref{eq:uu})的结果。
\section{研究成果的一些推导过程}
\subsection{两个未知节点协作最小误差界的一个充分条件}\label{B_F_0}
由式(\ref{eq:4_characteristic_polynomial}),SPEB为其所有根的倒数和,因此具有如下形式
\begin{equation}\label{eq:SPEB_GLOBAL}
\text{SPEB}=\frac{\displaystyle\sum \frac{1}{\lambda_i}+\epsilon(\frac{\cos^2(\theta)}{\lambda_1}(\sum_{i \neq 1}\frac{1}{\lambda_i})+\frac{\sin^2(\theta)}{\lambda_2}(\displaystyle\sum_{i \neq 2}\frac{1}{\lambda_i})+\frac{\cos^2(\phi)}{\lambda_3}(\sum_{i \neq 3}\frac{1}{\lambda_i})+\frac{\sin^2(\phi)}{\lambda_4}(\sum_{i \neq 4}\frac{1}{\lambda_i}))}{\displaystyle 1+\epsilon(\frac{\cos^2(\theta)}{\lambda_1}+\frac{\sin^2(\theta)}{\lambda_2}+\frac{\cos^2(\phi)}{\lambda_3}+\frac{\sin^2(\phi)}{\lambda_4})}.
\end{equation}
对固定的$\phi$,我们证明SPEB关于$\theta$是单调递减的。记$k=\frac{\cos^2 \theta}{a_1}+\frac{\sin^2 \theta}{a_2}$,$u=(\frac{1}{a_3}+\frac{1}{a_4}),v=(\frac{1}{a_1}+\frac{1}{a_2})$.~(\ref{eq:SPEB_GLOBAL})可以化为:
\begin{equation}
\text{SPEB}=u\frac{1+(\frac{1}{a_1}+\frac{1}{a_2})/u+\epsilon(\frac{1}{a_1a_2 u}+\frac{1}{a_3a_4 u}+k+(\frac{\cos^2\phi}{a_3}+\frac{\sin^2\phi}{a_4})\frac{v}{u})}{1+\epsilon(k+(\frac{\cos^2\phi}{a_3}+\frac{\sin^2\phi}{a_4}))}.
\end{equation}
SPEB可以写成关于k的反比例函数$u\frac{A+k}{B+k}$的形式,其中
\begin{align}\notag
A=&(1+(\frac{1}{a_1}+\frac{1}{a_2})/u)/\epsilon+\frac{1}{a_1a_2 u}+\frac{1}{a_3a_4 u}+(\frac{\cos^2\phi}{a_3}+\frac{\sin^2\phi}{a_4})\frac{v}{u}\\
B=&1/\epsilon+(\frac{\cos^2\phi}{a_3}+\frac{\sin^2\phi}{a_4}).
\end{align}
如果能证明$A \geq B$,那么该反比例函数关于k是单调递减的。
\begin{align}\notag
A-B=&(\frac{1}{a_1}+\frac{1}{a_2})/u\epsilon+\frac{1}{a_1a_2 u}+\frac{1}{a_3a_4 u}+(\frac{\cos^2\phi}{a_3}+\frac{\sin^2\phi}{a_4})(\frac{v}{u}-1)\\
\geq &\frac{1}{a_3a_4 u}+(\frac{\cos^2\phi}{a_3}+\frac{\sin^2\phi}{a_4})(\frac{v}{u}-1)\\
=&\frac{1}{u}((\frac{1}{a_1}+\frac{1}{a_2})(\frac{\cos^2\phi}{a_3}+\frac{\sin^2\phi}{a_4})-(\frac{\cos^2\phi}{a_3^2}+\frac{\sin^2\phi}{a_4^2})).\notag
\end{align}
由假设:$\frac{1}{a_1}+\frac{1}{a_2}\geq \max\{\frac{1}{a_4},\frac{1}{a_3}\}$,
所以$A-B\geq 0$。
\begin{equation}
k=\frac{1}{a_1}+\sin^2 \theta(\frac{1}{a_2}-\frac{1}{a_1}).
\end{equation}
根据上式,和假设条件$a_1\geq a_2$,可得在$\sin^2(\theta)\in[0,1]$的区间内k关于$\sin^2(\theta)$是单调递增的。所以由复合函数的单调性,SPEB关于$\theta$是单调递减的。
同理可证明条件$\frac{1}{a_3}+\frac{1}{a_4}\geq \max\{\frac{1}{a_1},\frac{1}{a_2}\}$是保证固定$\theta$的情况下SPEB关于$\phi \in [0,\frac{\pi}{2}]$是单调递减的。
\subsection{单节点动态定位问题等效费舍尔信息矩阵推导}\label{B_F_1}
为简化符号,记$\bm{u}:=\bm{u}_{N_a-1}$,2阶单位阵记为$\bm{I}_2$,由等效费舍尔信息矩阵的定义,有
\begin{align}\notag\label{eq:initial_efim}
  U_{N_a}=&\lambda \bm{I}_2+\bm{u}\bm{u}^{\textrm{T}}-\bm{u}\bm{u}^{\textrm{T}} U_{N_a-1}^{-1}\bm{u}\bm{u}^{\textrm{T}}\\
  =&\lambda \bm{I}_2+(1-\bm{u}^{\textrm{T}} U_{N_a-1}^{-1}\bm{u})\bm{u}\bm{u}^{\textrm{T}}.
\end{align}
因为$\bm{u}\bm{u}^{\textrm{T}}=U\begin{pmatrix}
                     1 & 0 \\
                     0 & 0
                   \end{pmatrix}U^{-1}$,其中$U$是由$\bm{u}$的方向角确定的二维旋转矩阵,所以
$U_{N_a}$相似于下面的矩阵
\begin{equation}
U_{N_a}\sim \begin{pmatrix}
                           \lambda+1-\bm{u}^{\textrm{T}} U_{N_a-1}^{-1}\bm{u} & 0 \\
                           0 & \lambda
                         \end{pmatrix}.
\end{equation}

我们定义$T_i=\lambda+1-\bm{u}^{\textrm{T}} U_{N_a-i}^{-1}\bm{u}$并设$U_{N_a-1}=\bm{u}\bm{u}^{\textrm{T}}+J_2$
由式(\ref{eq:woodbury})可得
\begin{align}\notag
T_1=&\lambda+1-\bm{u}^{\textrm{T}} (\bm{u}\bm{u}^{\textrm{T}}+J_2)^{-1}\bm{u}\\
=&\lambda+(1+\bm{u}^{\textrm{T}} J_2^{-1}\bm{u})^{-1}.
\end{align}
进一步设$v:=\bm{u}_{N_a-2}$,则$J_2=\lambda \bm{I}_2+(1-\bm{v}^{\textrm{T}} U_{N_a-2}^{-1}\bm{v})\bm{v}\bm{v}^{\textrm{T}}=V\begin{pmatrix}
                     \lambda+1-\bm{v}^{\textrm{T}} U_{N_a-2}^{-1}\bm{v} & 0 \\
                     0 & \lambda
                   \end{pmatrix}V^{-1}$
设$\bm{u}=(\cos\phi_1,\sin\phi_1)^{\textrm{T}},\bm{v}=(\cos\phi_2,\sin\phi_2)^{\textrm{T}}$,则
\begin{equation}
V^{-1}\bm{u}=\begin{pmatrix}
                     \cos\phi_2 & \sin\phi_2 \\
                     -\sin\phi_2 & \cos\phi_2
                   \end{pmatrix}\binom{\cos\phi_1}{\sin\phi_1}=\binom{\cos(\phi_1-\phi_2)}{\sin(\phi_1-\phi_2)}=:w.
\end{equation}
所以
\begin{align*}
T_1=&\lambda+(1+\bm{w}^{\textrm{T}} \begin{pmatrix}
                     \lambda+1-\bm{v}^{\textrm{T}} U_{N_a-2}^{-1}\bm{v} & 0 \\
                     0 & \lambda
                   \end{pmatrix}^{-1}\bm{w})^{-1}\\
                   =&\frac{1}{1+\cfrac{\cos^2(\phi_1-\phi_2)}{T_2}+\cfrac{\sin^2(\phi_1-\phi_2)}{\lambda}}.
\end{align*}
递推可得一般形式。
终止条件:
\begin{align*}
T_{N_a-1}&=\lambda+1-\bm{u}_1^{\textrm{T}}(\lambda\bm{I}+\bm{u}_1\bm{u}_1^{\textrm{T}})^{-1}\bm{u}_1\\
&=\lambda+\cfrac{1}{\lambda+\cfrac{1}{\lambda}}.
\end{align*}
\subsection{定理\ref{theorem:arbitrary_curve}的证明}\label{B_F_6}
式(\ref{eq:starting_or_ending})给出了式(\ref{eq:limiting_cf})右端是$\bm{p}(t)$为直线的情形。由于对于任意的平面曲线和角度序列$\{\theta_i\}$,$T_1(N_a)$是关于$N_a$的增函数且小于$\lambda+1$,因此式(\ref{eq:limiting_cf})左端的极限总是存在的。
考虑由$\bm{p}(t)$确定的角度序列$\{\theta_i\}$以如下的方式趋近于直线对应的直线序列:
\begin{equation}
\{\theta_1,\theta_2,\theta_3,\dots\}\rightarrow\{0,\theta_2,\theta_3,\dots\}\rightarrow
\{0,0,\theta_3,\dots\}\rightarrow\dots
\end{equation}
记将前n个角度置零后由式(\ref{eq:recursive_efim})确定的连分式为$K_n$,我们首先给出:
\begin{equation}\label{eq:arbitrary_curve_1}
\lim_{n\to\infty}K_n=\frac{\lambda+\sqrt{4\lambda+\lambda^2}}{2}
\end{equation}
为证式(\ref{eq:arbitrary_curve_1}),记角度序列$\{0,0,\dots,\theta_{n+1},\dots,\}$去掉前r项后对应的连分式为$K^r_n$
\begin{align}\notag
|K_n-M^*|=&\frac{|\frac{1}{M^*}-\frac{1}{K^2_n}|}{(1+\frac{1}{M^*})(1+\frac{1}{K^2_n})}\\
\leq &|\frac{1}{M^*}-\frac{1}{K^2_n}|\notag\\
= &\frac{|\frac{1}{M^*}-\frac{1}{K^3_n}|}{K^2_n(M^*+1)(1+\frac{1}{K^3_n})}\\
=& \frac{|\frac{1}{M^*}-\frac{1}{K^3_n}|}{(M^*+1)(\lambda+1+\frac{1}{K^3_n+1})}\notag\\
\leq & \frac{|\frac{1}{M^*}-\frac{1}{K^3_n}|}{(\lambda+1)^2}.\notag
\end{align}
当$r<n$ 时,
\begin{equation}
|\frac{1}{M^*}-\frac{1}{K^r_n}|\leq \frac{|\frac{1}{M^*}-\frac{1}{K^{r+1}_n}|}{(\lambda+1)^2}.
\end{equation}
因此:
\begin{equation}
|K_n-M^*|\leq \frac{|\frac{1}{M^*}-\frac{1}{K^{n}_n}|}{(\lambda+1)^{2(n-2)}}.
\end{equation}
故式(\ref{eq:arbitrary_curve_1})成立。
补充定义$K_0=\lim_{\Delta t\to 0}T_1(N_a)$这样式(\ref{eq:limiting_cf})即等价为
\begin{equation}\label{eq:equivalent_limiting_cf}
\sum_{i=1}^{\infty}(K_{i-1}-K_{i})=0.
\end{equation}
先考虑$K_0-K_1$,二者的差别是$\theta_1$是否为0,
\begin{align}\notag
K_0-K_1=&\frac{1}{1+\frac{1}{K_1^1}}-\frac{1}{1+\frac{1}{K_1^1}+\sin^2\theta_1(\frac{1}{\lambda}-\frac{1}{K_1^1})}\\
\leq & (\frac{1}{\lambda}-\frac{1}{K_1^1})\sin^2\theta_1.
\end{align}
类似式(\ref{eq:arbitrary_curve_1})的推导:
\begin{equation}
K_r-K_{r+1}\leq \sin^2\theta_{r+1}(\frac{1}{\lambda}-\frac{1}{K_{r+1}^{r+1}})\frac{1}{(\lambda+1)^{2r}}.
\end{equation}
由条件$\bm{p}'(t)$存在且连续可得切向量是连续变化的,由微分中值定理在闭区间内存在常数c使得角度变化量$\theta_i\leq c\Delta t$。
由正弦函数的单调性推出:
\begin{equation}
K_r-K_{r+1}\leq \sin^2 (c\Delta t) \frac{1}{\lambda}\frac{1}{(\lambda+1)^{2r}}.
\end{equation}
因此
\begin{equation}\label{eq:quatratic_convergence}
0\leq \sum_{i=1}^{N_a}(K_{i-1}-K_{i})\leq \sin^2 (c\Delta t) \frac{1}{\lambda}\sum_{i=1}^{\infty}\frac{1}{(\lambda+1)^{2i}}.
\end{equation}
无穷级数$\sum_{i=1}^{\infty}\frac{1}{(\lambda+1)^{2i}}$收敛,所以当$\Delta t\to 0$时$N_a\to \infty$,式(\ref{eq:equivalent_limiting_cf})成立。

%\subsection{单节点动态定位问题等效费舍尔信息衰减上下界}\label{B_F_2}
%为记号简便记$T'_1=T_1(N_a+1),T_1=T_1(N_a)$
%由于我们在连分式的第一层嵌入了新的节点,设引入的角度参数为$\theta$,于是可以得到:
%\begin{equation}
%T'_1=\lambda+\frac{1}{1+\frac{\sin^2\theta}{\lambda}+\frac{\cos^2\theta}{T_1}}.
%\end{equation}
%对固定的$T_1>\lambda$,$T'_1$是关于$\theta$的减函数,于是我们有:
%\begin{equation}
%T'_1\leq \lambda+\frac{1}{1+\frac{1}{T_1}}.
%\end{equation}
%递推下去我们可以得到结论$T_1$当各个角度的正弦值均为0时取得最大值:
%\begin{align}\label{eq:MMM}
%T_1\leq &[\lambda,1,\underbrace{\lambda,1}_{\text{repeat $N_a-1$ times}},\lambda]:=M_{N_a}\\
%\leq  & M^* \text{参见式}(\ref{eq:starting_or_ending})\notag
%\end{align}
%下面考虑$T'_1-T_1$:
%\begin{equation}
%T'_1-T_1\leq \lambda-\frac{T_1^2}{T_1+1}
%\end{equation}
%且上式右端关于$T_1$是减函数,故由式(\ref{eq:MMM})可得:
%\begin{equation}
%T'_1-T_1\leq \lambda-\frac{T_1^2}{M_{N_a}+1}
%\end{equation}

%考虑在原有基础上增加一层节点,
%于是协作层数由原来的$N_a-1$变为$N_a$。等效费舍尔信息矩阵较大的特征值分别记为$T_1,T'_1$,

%上界:
\subsection{单节点动态定位问题等效费舍尔信息衰减上下界}\label{B_F_2}
为记号简便记$T'_1=T_1(N_a+1),T_1=T_1(N_a)$
%考虑在原有基础上增加一层节点,
%于是协作层数由原来的$N_a-1$变为$N_a$。等效费舍尔信息矩阵较大的特征值分别记为$T_1,T'_1$,
为便于比较,我们在$T_1$中引入虚拟节点将其层数也拓展为$N_a$,它只有锚点的定位信息,这样它们的区别是连分式的末端$T_{N_a}=\lambda$,$T'_{N_a}=\lambda+\frac{1}{1+1/\lambda}$
对$|T_1-T'_1|$从外向里通分得:
\begin{align}\notag
|T_1-T'_1|=&\frac{|\frac{1}{T_2}-\frac{1}{T'_2}|\cos^2\theta_1}{(1+\frac{\sin^2\theta_1}{\lambda}+\frac{\cos^2\theta_1}{T_2})
(1+\frac{\sin^2\theta_1}{\lambda}+\frac{\cos^2\theta_1}{T'_2})}\\
\leq &|\frac{1}{T_2}-\frac{1}{T'_2}|.
\end{align}
继续放缩$|\frac{1}{T_2}-\frac{1}{T'_2}|$有:
\begin{align}\notag
|\frac{1}{T_2}-\frac{1}{T'_2}|= & \frac{|\frac{1}{T_3}-\frac{1}{T'_3}|\cos^2\theta_2}{(1+\lambda(1+\frac{\sin^2\theta_2}{\lambda}+\frac{\cos^2\theta_2}{T_3}))
(1+\lambda(1+\frac{\sin^2\theta_2}{\lambda}+\frac{\cos^2\theta_2}{T'_3}))}\\
\leq & |\frac{1}{T_3}-\frac{1}{T'_3}|\frac{1}{(1+\lambda)^2}.
\end{align}
逐次递推得
\begin{equation}
|T_1-T'_1|\leq \frac{1}{(1+\lambda)^{2(N_a-2)}} |\frac{1}{T_{N_a}}-\frac{1}{T'_{N_a}}|.
\end{equation}
而:
\begin{equation}
|\frac{1}{T_{N_a}}-\frac{1}{T'_{N_a}}|=\frac{1}{\lambda^2+2\lambda}.
\end{equation}
对于下界,因为$T_2,T'_2\geq \lambda$
\begin{equation}
|T_1-T'_1|\geq \frac{\cos^2\Delta\theta|\frac{1}{T_2}-\frac{1}{T'_2}|}{(1+1/\lambda)^2}.
\end{equation}
\begin{equation}
|\frac{1}{T_2}-\frac{1}{T'_2}|\geq \frac{\cos^2\Delta\theta|\frac{1}{T_3}-\frac{1}{T'_3}|}{(2+\lambda)^2}
\end{equation}
逐次递推得下界。
%\subsection{推论\ref{corollary:exponential_decreasing}的证明}\label{B_F_5}
%由定理\ref{theorem:exponential_decreasing}的结论:
%\begin{equation}
%T_{N_a}=\sum_{k=1}^{N_a-1}\Delta_{+}T_k
%\end{equation}
%因此当$N_a\geq 2$时
%\begin{equation}
%|T_{N_a}-T_{\infty}|=\sum_{k=N_a}^{\infty}\Delta_{+}T_k\leq %\frac{1}{\lambda^2+2\lambda}\sum_{k=N_a}^{\infty}\frac{1}{(\lambda+1)^{2(k-2)}}=\frac{1}{(\lambda^2+2\lambda)^2}\frac{1}{(\lambda+1)^{2(N_a-3)}}
%\end{equation}
%取$q=(\lambda+1)^2$,其余常数为C即可。
\subsection{单节点非均一测距误差等效费舍尔信息矩阵推导}\label{B_F_3}
类似式(\ref{eq:initial_efim})有:
\begin{equation}
U_{N_a}=\bm{I}+(\lambda_1-\lambda_1^2 \bm{u}^{\textrm{T}} U_{N_a-1}^{-1}\bm{u})\bm{u}\bm{u}^{\textrm{T}}.
\end{equation}
而
\begin{equation}
T_1=1+\lambda_1-\lambda_1^2 \bm{u}^{\textrm{T}} U_{N_a-1}^{-1}\bm{u}=1+(\lambda_1^{-1}+\bm{u}^{\textrm{T}}\bm{J}_2^{-1}\bm{u})^{-1}.
\end{equation}
对$J_2=\bm{I}_2+(\lambda_2-\lambda_2^2\bm{v}^{\textrm{T}} U_{N_a-2}^{-1}\bm{v})\bm{v}\bm{v}^{\textrm{T}}$提取关于$\bm{v}$的旋转矩阵即得到
式(\ref{eq:recursive_efim_second})。
另外从式(\ref{eq:recursive_efim_second})出发,设$\theta'_1\leq \theta_1$,作差:
\begin{equation}
T_1(\theta'_1)-T_1(\theta_1)=\frac{1}{A}(\cos^2\theta'_1-\cos^2\theta_1)(1-\frac{1}{T_2})\geq 0
\end{equation}
其中$A>0$,由上式可看出角度$\theta_1$越小$T_1$越大,同理$\theta_i$越小$T_i$越大,而由连分式的表达式可以看出$T_1$关于$T_2$递增,而连分式本身具有自相似性,因此诸$\theta_i$减小可以增大信息量$T_1$。

\subsection{引理~\ref{lemma:hexagon}的推导}\label{B_F_4}
  设式(\ref{eq:equiv})左边为B,$A=\lambda+\frac{3}{2}$,那么由Woodbury矩阵求逆公式有
  \begin{equation}
  (A+\frac{1}{\lambda+3/2-\frac{3/2}{\lambda+3/2}})^{-1}=A^{-1}-A^{-1}BA^{-1}.
  \end{equation}
  整理得:
  \begin{equation}
  B=A-\frac{A^2}{A+\frac{1}{\lambda+3/2-\frac{3/2}{\lambda+3/2}}}.
  \end{equation}
  通分化简得证。
\section{本文中用到的关于连分式的结论}\label{C_F}
\begin{definition}
  有限序列$t_1,t_2,\dots,t_r$满足$t_j\geq 1$对于$j\geq2$可以递推地定义有限连分式
  \begin{equation}
  [t_1,t_2,\dots,t_r]:=t_1+\frac{1} {[t_2,\dots,t_r]}.
  \end{equation}
\end{definition}
\begin{theorem}\label{thm:basic}
  设$p_j=t_j p_{j-1}+p_{j-2},q_j=t_j q_{j-1}+q_{j-2}$,$M_j=\left(\begin{matrix}p_j&p_{j-1}\\q_j&q_{j-1}\end{matrix}\right)$,
  $p_0,p_1,q_0,q_1$由$M_0=I_2$给出,且$T_j=\left(\begin{matrix}t_j&1\\1&0\end{matrix}\right)$则有下面三个恒等式:
  \begin{enumerate}
    \item $M_j=M_{j-1}T_j$,
    \item $\binom{p_j}{q_j}=(\prod_{i=1}^r T_i )\binom{1}{0}$,
    \item $[t_1,t_2,\dots,t_r]=\frac{p_r}{q_r}$.
  \end{enumerate}
\end{theorem}
\begin{theorem}\label{theorem:quadratic_cyclic}
若$\lim_{r\to \infty}[t_1,t_2,\dots,t_r]$存在,该极限是形如$\frac{a+b\sqrt{m}}{c}$的二次根式当且仅当序列$t_i$从某项开始是周期的,即$\exists c\text{和}r,\,s.t.\, t_{i}=t_{i+r},\forall i\geq c$。
\end{theorem}
\begin{remark}

对于定理\ref{theorem:quadratic_cyclic},$\lim_{r\to \infty}[t_1,t_2,\dots,t_r]$值可用二次方程不动点的方法求出\cite{ContinuedFraction}。
%设$\left(\begin{matrix}a&b\\c&d\end{matrix}\right)=(\prod_{i=1}^{rc} %T_{i+c})$,称$\left(\begin{matrix}a&b\\c&d\end{matrix}\right)$对应着分式线性变换K,如果$K(z)=\frac{az+b}{cz+d}$。
%可以证明系数为实数的分式线性变换以函数复合作为群运算与行列式为1的2阶方阵以矩阵乘积作为群运算是同构的。
%c之前的$t_i$对应着矩阵$T_i$的乘积矩阵看作分式线性变换F。可以进一步说明二次根式和T和F的有如下的关系:
%\begin{enumerate}
%  \item 首先求解$K$的不动点即解二次方程$x=\frac{ax+b}{cx+d}$得x
%  \item $F(x)=\lim_{r\to \infty}[t_1,t_2,\dots,t_r]$,F(x)即为极限$\lim_{r\to \infty}[t_1,t_2,\dots,t_r]$
%\end{enumerate}
%一般二次方程有两个根,而有限连分式的极限值是唯一的,这时可根据极限值介于序列的前两个数之间剔除一个不合理的不动点。
\end{remark}
