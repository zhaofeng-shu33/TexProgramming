% \documentclass[10pt,conference]{IEEEtran}
% \usepackage{xeCJK}%preamble part
% \usepackage{graphicx}
% \usepackage{indentfirst}
% \usepackage{enumerate}
% \usepackage[a4paper, inner=1.5cm, outer=3cm, top=2cm, bottom=3cm, bindingoffset=1cm]{geometry}
% \usepackage{epstopdf}
% \usepackage{listings}
% \usepackage{multicol}
% \usepackage{array}
% \usepackage{fontspec}
% \usepackage{bm}
% \usepackage{gensymb}
% \usepackage{todonotes}
% \usepackage{amsmath, amsthm, amssymb}
% \usepackage[citecolor=blue]{hyperref}
% \newtheorem{definition}{Definition}
% \newtheorem{thm}{Theorem}[section]
% \newtheorem{thm}{Theorem}
% \newtheorem{cor}{Corollary}
% \newtheorem{proposition}{Proposition}
% \newtheorem{lem}{Lemma}
% \newtheorem{remark}{Remark}
% \DeclareMathOperator{\sgn}{sgn}
% \theoremstyle{remark}
% \newtheorem*{rem}{Remark}
% \setCJKmainfont[BoldFont={SimHei}]{SimSun}
% \setCJKmonofont{SimSun}
% \setmainfont{Times New Roman}
% \newCJKfontfamily[hei]\heiti{SimHei}
% \setlength{\extrarowheight}{4pt}
% \setlength{\parindent}{1cm}
 
% \begin{document}
% \onecolumn
% \title{宽带定位的理论极限--第二部分:协作网络} 
% \author{\fontsize{12pt}{\baselineskip}{沈渊,清华大学电子工程系副教授}}
% \maketitle
% \begin{multicols}{2}
% \begin{abstract}
\title{第三篇--宽带定位的理论极限---第二部分:协作网络(节选)$^{[3]}$}
%第三篇论文的翻译,原文的题目是
%Fundamental Limits Of Wideband Localization--Part II: Cooperative Networks
%由于原文较长,翻译部分只针对原文关于EFIM的一阶上下界的部分。
%\end{abstract}


对于每个移动节点,EFIM的准确表达式非常复杂。但是我们可以找到每个节点EFIM的上下界,从而获得对定位问题的洞见。

\begin{proposition}
设$\bm{J}_e^A(\bm{p}_k)=\bm{F}(\mu_k,\eta_k,\vartheta_k)$表示节点k从锚点获得的定位信息,$\bm{C}_{kj}=\bm{F}(\nu_{kj},0,\phi_{kj})$表示该节点和节点j协作的测距信息RI。节点k的EFIM $\bm{J}_e(\bm{p}_k)$满足如下不等式:
\[
\bm{J}_e^L(\bm{p}_k) \preceq \bm{J}_e(\bm{p}_k) \preceq \bm{J}_e^U(\bm{p}_k)
\]
其中
\begin{eqnarray}
\bm{J}_e^L(\bm{p}_k)=\bm{J}_e^A(\bm{p}_k)+\sum_{j\in \mathcal{N}_a \backslash {k}} \xi_{kj}^L \bm{C}_{kj}\\
\bm{J}_e^U(\bm{p}_k)=\bm{J}_e^A(\bm{p}_k)+\sum_{j\in \mathcal{N}_a \backslash {k}} \xi_{kj}^U \bm{C}_{kj}
\end{eqnarray}
\end{proposition}
%\end{multicols}
%\hrulefill
\begin{proof}
不失一般性,我们假设k=1.

下界:考虑EFIM$\bm{J}_e^L(\bm{P})$为:
\begin{equation}
\bm{J}_e^L(\bm{P})=\left[
\begin{array}{cccc}
\bm{J}_e^A(\bm{p}_1)+\sum_{j\in \mathcal{N}_a\backslash \{1\}}\bm{C}_{1,j}&-\bm{C}_{1,2}& \dots & -\bm{C}_{1,N_a}\\
-\bm{C}_{1,2} &\bm{J}_e^A(\bm{p}_2)+\bm{C}_{1,2} & \dots & 0\\
\vdots & \vdots & \ddots & \vdots\\
-\bm{C}_{1,N_a} & 0 & \dots & \bm{J}_e^A(\bm{p}_{N_a})+\bm{C}_{1,N_a}
\end{array}
\right]
\end{equation}
%\hrulefill
%\begin{multicols}{2}
这个矩阵是令$\bm{J}_e(\bm{P})$中所有的$\bm{C}_{kj}=0$,对于$1\le k,j \leq N_a$.这个EFIM对应着节点2到$N_a$的协作完全被忽略。通过线性代数的指数可以证明$\bm{J}_e^L(\bm{P})\preceq \bm{J}_e^L(\bm{P})$,这也可直观相符,因为没有利用节点2到$N_a$的协作信息。利用EFI的方法,我们可以得到节点1的EFIM是:
\begin{equation*}
\begin{split}
&\bm{J}_e^L(\bm{p}_1)=\bm{J}_e^A(\bm{p}_1)\\
&+\sum_{j\in \mathcal{N}_a \backslash \{1\}}[\bm{C}_{1,j}-\bm{C}_{1,j}(\bm{J}_e^A(\bm{p}_j)+\bm{C}_{1,j})^{-1}\bm{C}_{1,j}]
\end{split}
\end{equation*}
因为$\bm{C}_{1,j}=\nu_{1,j}\bm{q}_{\phi_{1,j}}\bm{q}_{\phi_{1,j}}^T$,其中$\bm{q}_{\phi_{1,j}}\triangleq=[\cos(\phi_{1,j}),\sin(\phi_{1,j})]^T$,我们可以将$\bm{J}^L_e(\bm{p}_1)$表示成:
\begin{equation}
\bm{J}_e^L(\bm{p}_1)=\bm{J}_e^A(\bm{p}_1)
+\sum_{j\in \mathcal{N}_a \backslash \{1\}}\xi_{1,j}^L\bm{C}_{1,j}
\end{equation}
其中$\xi_{1,j}^L\triangleq 1-\nu_{1,j}\bm{q}^T_{\phi_{1,j}}(\bm{J}^A_e(\bm{p}_j)+\bm{C}_{1,j})^{-1}\bm{q}_{\phi_{1,j}}$.系数$\xi_{1,j}^L$可以进一步化简为
\begin{equation}
\begin{split}
\xi_{1,j}^L=&1-\nu_{1,j}\bm{q}^T_{\vartheta_j-\phi_{1,j}}\\
&\cdot (\text{diag}\{\mu_j,\eta_j\}+\nu_{1,j}\bm{q}_{\vartheta_j-\phi_{1,j}}\bm{q}^T_{\vartheta_j-\phi_{1,j}})^{-1}\bm{q}_{\vartheta_j-\phi_{1,j}}\\
=&\frac{1}{1+\nu_{1,j}\delta_j(\phi_{1,j})}
\end{split}
\end{equation}
其中
$$\delta_j(\phi_{1,j})=\frac{1}{\mu_j}\cos^2(\vartheta-\phi_{1,j})+\frac{1}{\eta_j}\sin^2(\vartheta-\phi_{1,j})$$

上界: 考虑EFIM$\bm{J}_e^U(\bm{P})$为:
%\end{multicols}
%\hrulefill
\begin{equation}
\bm{J}_e^U(\bm{P})=\left[
\begin{array}{cccc}
\bm{J}_e^A(\bm{p}_1)+&-\bm{C}_{1,2}& \dots & -\bm{C}_{1,N_a}\\
\sum_{j\in \mathcal{N}_a\backslash \{1\}}\bm{C}_{1,j}&&&\\
-\bm{C}_{1,2} &\bm{J}_e^A(\bm{p}_2)& \dots & 0\\
&+\bm{C}_{1,2}+\sum_{j\in \mathcal{N}_a\backslash \{1,2\}}2\bm{C}_{1,j} &&\\
\vdots & \vdots & \ddots & \vdots\\
-\bm{C}_{1,N_a} & 0 & \dots & \bm{J}_e^A(\bm{p}_{N_a})\\
&&&+\sum_{j\in \mathcal{N}_a\backslash \{1\}}\bm{C}_{1,j}2\bm{C}_{1,N_a}\\
\end{array}
\right]
\end{equation}
%\hrulefill
%\begin{multicols}{2}
这个矩阵可以通过把$\bm{J}_e^(\bm{P})$种对角元$\bm{C}_{kj}$扩大一倍同时让非对角元$-\bm{C}_{kj}=0$,对$1\le k,j\leq N_a$.用线性代数的知识我们可以证明$\bm{J}_e^U(\bm{P}) \succeq \bm{J}_e(\bm{P})$,这也符合直观,因为节点2到$N_a$之间有了更多的协作。利用EFI的方法,我们可以得到节点1的EFIM是:
\begin{equation*}
\bm{J}_e^U(\bm{p}_1)=\bm{J}_e^A(\bm{p}_1)
+\sum_{j\in \mathcal{N}_a \backslash \{1\}}\xi_{1,j}^U\bm{C}_{1,j}
\end{equation*}
其中
\begin{equation}
\xi_{1,j}^U =\frac{1}{1+\nu_{1,j}\tilde{\delta}_j(\phi_{1,j})}
\end{equation}
而$$\tilde{\delta}_j(\phi_{1,j})=\frac{1}{\tilde{\mu}_j}\cos^2(\tilde{\vartheta}-\phi_{1,j})+\frac{1}{\tilde{\eta}_j}\sin^2(\tilde{\vartheta}-\phi_{1,j})$$
而$\tilde{\mu}_j,\tilde{\eta}_j,\tilde{\vartheta}_j$满足:
$$
\bm{F}(\tilde{\mu}_j,\tilde{\eta}_j,\tilde{\vartheta}_j)=\bm{J}^A_e(\bm{p}_j)+\sum_{k\in \mathcal{N}_a\backslash \{1,j\}} 2\bm{C}_{jk}
$$
%\end{multicols}
\end{proof}
%\end{document}